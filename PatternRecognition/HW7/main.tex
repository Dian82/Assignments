\documentclass[reportComp]{thesis}
\usepackage[python,pseudo]{mypackage}

\title{模式识别作业Chap 6}
\school{数据科学与计算机学院}
\author{陈鸿峥}
\classname{17大数据与人工智能}
\stunum{17341015}
\headercontext{模式识别作业}

\begin{document}

\maketitle

\begin{question}[\textsection 6 Q3]
考虑用$n$个模式进行$m_e$次训练的一个$d-n_H-c$型网络。
\begin{itemize}
	\item [(a)] 此问题的空间复杂度是多少?(网络参数的存贮和模式存贮都要考虑,但不考虑程序本身)。
	\item [(b)] 假设网络训练由一个随机模式来训练,时间复杂度是多少?由于它受累计乘法次数的控制所以将此作为时间复杂度的测度。
	\item [(c)] 假设网络由成批模式训练,时间复杂度是多少?
\end{itemize}
\end{question}
\begin{answer}
\begin{itemize}
	\item [(a)] 隐含层$n_H$个神经元每个对应$d$个输入单元,因此有$n_Hd$个权重;同理,连接隐含层和输出层共$n_Hc$个权重,故一共的空间复杂度为$n_H(c+d)$。而模式存储的空间复杂度为$nd$。

	\item [(b)] 随机模式即计算
	\[\vw(t+1)=\vw(t)+\Delta\vw(t)\]
	先考虑隐含层到输出层,由公式(16),有
	\[\Delta w_{j k}=\eta\left(t_{k}-z_{k}\right) f^{\prime}\left(net_{k}\right) y_{j}\]
	其中,$net_k=\sum_{j=1}^{n_H}w_{jk}y_j$由$n_H$次乘法和$n_H$次加法构成,因此$\Delta w_{jk}$时间复杂度为$c(2n_H+1)$。

	再考虑输入层到隐含层,由公式(20),有
	\[\Delta w_{j i}=\eta x_{i} f^{\prime}\left(net_{j}\right) \sum_{k=1}^{c} w_{k j} \delta_{k}\]
	同上理,$net_j$计算量为$2d$,$\sum_{k=1}^{c} w_{k j} \delta_{k}$计算量为$2c$,故$\Delta w_{ji}$时间复杂度为$2n_H(d+c+1)$。

	进而,在一轮迭代中$\vw$的计算量为($\Delta \vw$的更新及$\Delta\vw$加上$\vw$的计算)
	\[c(2n_H+1)+2n_H(d+c+1)+(dn_H+n_Hc)=3dn_H+5n_Hc+c+2n_H\]
	得到总的时间复杂度为$(3dn_H+5n_Hc+c+2n_H)m_e$。

	\item [(c)] 同上理,总的模式迭代次数为$nm_e$,故时间复杂度为$(3dn_H+5n_Hc+c+2n_H)n m_e$。
\end{itemize}
\end{answer}

\begin{question}[\textsection 6 Q8]
考虑具有$d$个输入单元、$n_H$个隐单元、$c$个输出单元以及偏置的一个标准三层反向传播网。
\begin{itemize}
	\item [(a)] 网络中有多少权值?
	\item [(b)] 考虑权值对称。特别是,证明如果将每一个权值的符号反向,网络功能不变。
	\item [(c)] 现在考虑隐单元的对称交换。隐单元上没有标记,因此它们可以相互交换(沿着对应权值)
	而使网络功能不受影响。证明该等价标记数(对称交换因子)为$n_H!2^{n_H}$。在$n_H=10$的情况下估计该因子的值。
\end{itemize}
\end{question}
\begin{answer}
\begin{itemize}
	\item [(a)] 同\textsection 6 Q3(a)有权重$dn_H+(n_H+1)c$,其中多出来的$+1$项为偏置项。
	\item [(b)] 由公式(6)
	\[g_{k}(\mathrm{x}) \equiv z_{k}=f\left(\sum_{j=1}^{n_{H}} w_{k j} f\left(\sum_{i=1}^{d} w_{j i} x_{i}+w_{j 0}\right)+w_{k 0}\right)\]
	假设激活函数$f$为奇函数,则内层$f$在权值符号反向后,值也相反;但$f$前面还有一个权重项$w_{kj}$也经过了反向,故原值不变,即
	\[\left(-w_{k j}\right)\left[-f\left(\sum_{j=1}^{d} w_{j i} x_{i}\right)\right]=\left(w_{k j}\right)\left[f\left(\sum_{j=1}^{d} w_{j i} x_{i}\right)\right]\]
	\item [(c)] 考虑$n_H$个隐单元构成的集合的子集,一共有$2^{n_H}$个。
	而这些子集内的隐单元都可以进行重排,因此有$n_H!$种情况。
	故一共有$n_H!2^{n_H}$个对称交换因子。
	对于$n=10$,该值为$3715891200$。
\end{itemize}
\end{answer}

\begin{question}[\textsection 6 Q10]
在如下两种情况下,将sigmoid的导数用sigmoid本身来表示($a,b>0$):
\begin{itemize}
	\item [(a)] 完全为正的sigmoid:$f(net)=\frac{1}{1+\ee^{a\cdot net}}$。
	\item [(b)] 反对称的sigmoid:$f(net)=a\tanh(b\cdot net)$
\end{itemize}
\end{question}
\begin{answer}
\begin{itemize}
	\item [(a)] 变换得$1+\ee^{a\cdot net}=1/f(net)$,进而
	\[f'(net)=\frac{-a\ee^{a\cdot net}}{(1+\ee^{a\cdot net})^2}=-a(f(net)-f^2(net))\]
	\item [(b)] 变换得$\tanh(b\cdot net)=f(net)/a$,进而
	\[f'(net)=ab\mathrm{sech}^2(b\cdot net)=ab(1-\tanh^2(b\cdot net))=ab(1-f^2(net)/a^2)\]
\end{itemize}
\end{answer}

\begin{question}[\textsection 6 Q12]
解释为什么输入层到隐含层的权值必须相互不等(即是随机的),否则学习不能顺利进行。
更明确地说,如果权值初始化为相同的值,将出现什么现象?
\end{question}
\begin{answer}
若输入层到隐层的权值均为$w_o$,则
\[net_{j}=f\left(net_{j}\right)=\sum_{i=1}^{d} w_{ji} x_{i}=w_{o} \sum_{i} x_{i}=w_{o} \mathbf{x}\]
为常数,因此梯度不会产生变化,进而无法训练。
\end{answer}

\begin{question}[\textsection 6 Q17]
完成导出式(26)的推导步骤
\end{question}
\begin{answer}
由式(25)有
\[J(\mathbf{w})=n\left[\frac{n_{k}}{n} \frac{1}{n_{k}} \sum_{\mathbf{x} \in \omega_{k}}\left[g_{k}(\mathbf{x}, \mathbf{w})-1\right]^{2}+\frac{n-n_{k}}{n} \frac{1}{n-n_{k}} \sum_{\mathbf{x} \notin\omega_{k}} g_{k}(\mathbf{x}, \mathbf{w})^{2}\right]\]
当$n\to\infty$时,$\omega_k$中的样本比例趋于$P(\omega_k)$,由大数定律
\[\frac{1}{n_{k}} \sum_{\mathbf{x} \in \omega_{k}}\left[g_{k}(\mathbf{x}, \mathbf{w})-1\right]^{2}\]
趋于
\[\mathbb{E}\left(\left[g_{k}(\mathbf{x}, \mathbf{w})-1\right]^{2} | \mathbf{x} \in \omega_{k}\right)=
\int\left[g_{k}(\mathbf{x}, \mathbf{w})-1\right]^{2} p\left(\mathbf{x} | \omega_{k}\right) \diff \vx\]
类似的有
\[\frac{1}{n-n_k}\sum_{x\notin \omega_k}[F_k(\vx,\omega)]^2\]
趋于
\[\mathbb{E}\left([g_k(\vx,\vw)]^2\mid\vx\in\omega_{i\ne k}\right)
=\int[g_k(\vx,\vw)]^2p(\vx\mid\omega_{i\ne k})\diff\vx\]
进而得到
\[\begin{aligned}
J(\mathbf{w}) &=P\left(\omega_{k}\right) \int\left[g_{k}(\mathbf{x}, \mathbf{w})-1\right]^{2} p\left(\mathbf{x} | \omega_{k}\right) d \mathbf{x}+P\left(\omega_{i \neq k}\right) \int\left[g_{k}(\mathbf{x}, \mathbf{w})\right]^{2} p\left(\mathbf{x} | \omega_{i \neq k}\right) d \mathbf{x} \\
&=\int\left[g_{k}(\mathbf{x}, \mathbf{w})-1\right]^{2} p\left(\mathbf{x}, \omega_{k}\right) d \mathbf{x}+\int\left[g_{k}(\mathbf{x}, \mathbf{w})\right]^{2} p\left(\mathbf{x} | \omega_{i \neq k}\right) d \mathbf{x} \\
&=\int\left[g_{k}^{2}(\mathbf{x}, \mathbf{w})+1-2 g_{k}(\mathbf{x}, \mathbf{w})\right] p\left(\mathbf{x}, \mathbf{w}_{k}\right) d \mathbf{x}+\int\left[g_{k}(\mathbf{x}, \mathbf{w})\right]^{2} p\left(\mathbf{x} | \omega_{i \neq k}\right) d \mathbf{x} \\
&=\int\left[g_{k}(\mathbf{x}, \mathbf{w})-P\left(\omega_{k} | \mathbf{x}\right)\right]^{2} p(\mathbf{x}) d \mathbf{x}+\int P\left(\omega_{k} | \mathbf{x}\right)\left[1-P\left(\omega_{k} | \mathbf{x}\right)\right] p(\mathbf{x}) d \mathbf{x} \\
&=\int\left[g_{k}(\mathbf{x}, \mathbf{w})-P\left(\omega_{k} | \mathbf{x}\right)\right]^{2} p(\mathbf{x}) d x+\int P\left(\omega_{k} | \mathbf{x}\right) P\left(\omega_{i \neq k} | \mathbf{x}\right) p(\mathbf{x}) d \mathbf{x}
\end{aligned}\]
\end{answer}

\begin{question}[\textsection 6 Q26]
考虑$S$型激活函数
\[f(net) = a \tanh (b\cdot net)
= a\left[\frac{1-\ee^{-b \cdot net}}{1+\ee^{-b\cdot net}}\right]
= \frac{2 a}{1+\ee^{-b\cdot net}}-a\]
\begin{itemize}
	\item [(a)] 证明它的导数$f'(net)$可简单写成$f(net)$的形式
	\item [(b)] 在$net=-\infty$、$0$、$+\infty$时$f(net)$、$f'(net)$、$f''(net)$分别是多少?
\end{itemize}
\end{question}
\begin{answer}
\textcolor{red}{注意原题的等式就是有问题的,$\tanh(b\cdot net)$展开的那条等式并不成立!
因此下面是按照最右侧的等式,即$f(net)=\frac{2 a}{1+\ee^{-b\cdot net}}-a$进行计算。}
\begin{itemize}
	\item [(a)] 同题\textsection 6 Q10(b),有
	\[f'(net)=\frac{2ab\ee^{b\cdot net}}{(1+\ee^{b\cdot net})^2}=\frac{b}{2a}[a^2-(f(net))^2]\]
	\item [(b)] 注意到$f''(net)=-\frac{b}{a}f(net)f'(net)$,故
	\begin{itemize}
		\item 当$net=\infty$时,有
		\[\begin{aligned}
		f(\infty) &=\frac{2 a}{1+\ee^{-b\cdot n e t}}-a=2 a-a=a \\
		f^{\prime}(\infty) &=\frac{b}{2 a}\left[a^{2}-(f(n e t))^{2}\right]=\frac{b}{2 a}\left(a^{2}-a^{2}\right)=0 \\
		f^{\prime \prime}(\infty) &=-\frac{b}{2 a} f(n e t) f^{\prime}(n e t)=0
		\end{aligned}\]
		\item 当$net=0$时,有
		\[\begin{aligned}
		f(0) &=\frac{2 a}{1+\ee^{-b\cdot n e t}}-a=2 a/2-a=a \\
		f^{\prime}(0) &=\frac{b}{2 a}\left[a^{2}-(f(n e t))^{2}\right]=\frac{b}{2 a}a^2=\frac{ab}{2} \\
		f^{\prime \prime}(0) &= 0
		\end{aligned}\]
		\item 当$net=-\infty$时,有
		\[\begin{aligned}
		f(-\infty) &=\frac{2 a}{1+\ee^{-b\cdot n e t}}-a=0-a=-a \\
		f^{\prime}(-\infty) &=\frac{b}{2 a}\left[a^{2}-(f(n e t))^{2}\right]=\frac{b}{2 a}\left(a^{2}-a^{2}\right)=0 \\
		f^{\prime \prime}(-\infty) &=0
		\end{aligned}\]
	\end{itemize}
\end{itemize}
\end{answer}

\end{document}