\documentclass[reportComp]{thesis}
\usepackage[cpp,pseudo]{mypackage}

\title{模式识别作业四}
\subtitle{}
\school{数据科学与计算机学院}
\author{陈鸿峥}
\classname{17大数据与人工智能}
\stunum{17341015}
\headercontext{模式识别作业}
\lstset{language=python}

\begin{document}

\maketitle

\begin{question}[\textsection 3 Q4]
设$\vx$为一个$d$维的二值向量,服从多维伯努利分布
\[P(\vx\mid\vtheta)=\prod_{i=1}^d\theta_i^{x_i}(1-\theta)^{1-x_i}\]
其中$\vtheta=(\theta_1,\ldots,\theta_d)^\T$为一个未知的参数向量,而$\theta_i$为$x_i=1$的概率。
证明:对于$\vtheta$的最大似然估计为
\[\hat{\vtheta}=\frac{1}{n}\sum_{k=1}^n\vx_k\]
\end{question}
\begin{answer}
考虑$n$个样本的采样$\{\vx_k\}_{k=1}^{n}$,似然函数为
\[L(\vtheta)=\prod_{k=1}^n\prod_{i=1}^d\theta_i^{x_{ki}}(1-\theta_i)^{1-x_{ki}}\]
对数似然函数为
\[\ell(\vtheta)=\sum_{k=1}^n\sum_{i=1}^dx_{ki}\lrp{\ln\theta_i+(1-x_{ki})\ln(1-\theta_i)}\]
对上式求梯度有
\[[\nabla_{\vtheta}\ell(\vtheta)]_i=\nabla_{\theta_i}\ell(\theta)=\frac{1}{\theta_i}\sum_{k=1}^nx_{ki}-\frac{1}{1-\theta_i}\sum_{k=1}^n(1-x_{ki})=0\]
对上式整理可得
\[\hat{\theta}_i=\frac{1}{n}\sum_{k=1}^nx_{ki}\]
表为向量形式即得结果。
\end{answer}

\begin{question}[\textsection 3 Q19]
假设我们有一组训练样本,都服从高斯分布,其协方差矩阵$\Sigma$已知,而均值$\vmu$未知。
进一步假设这个均值$\vmu$本身是随机取值的,服从均值为$\vm_0$,协方差为$\Sigma_0$的高斯分布。
\begin{itemize}
	\item [(a)] 均值$\vmu$的MAP估计是什么?
	\item [(b)] 假设我们用线性变换来变换坐标$\vx'=A\vx$,其中$A$为非奇异矩阵。
	那么,MAP能够对变换以后的$\vmu'$做出正确的估计吗?并加以解释。
\end{itemize}
\end{question}
\begin{answer}
\begin{itemize}
	\item [(a)] 对于高斯分布有对数似然函数$\ell(\vmu)$和概率密度函数$p(\vmu)$
	\[\begin{aligned}
	\ell(\vmu)
	&=\sum_{k=1}^{n} \ln \left[p\left(\mathbf{x}_{k} | \boldsymbol{\mu}\right)\right] \\
	&=-\frac{n}{2} \ln \left[(2 \pi)^{d}|\Sigma|\right]-\sum_{k=1}^{n} \frac{1}{2}\left(\mathbf{x}_{k}-\boldsymbol{\mu}\right)^{\T} \Sigma^{-1}\left(\mathbf{x}_{k}-\boldsymbol{\mu}\right)\\
	p(\boldsymbol{\mu})&=\frac{1}{(2 \pi)^{d / 2}\left|\Sigma_{0}\right|^{1 / 2}} \exp \left[-\frac{1}{2}\left(\boldsymbol{\mu}-\mathbf{m}_{0}\right)^{\T} \Sigma_{0}^{-1}\left(\boldsymbol{\mu}-\mathbf{m}_{0}\right)\right]
	\end{aligned}\]
	进而MAP估计为
	\[\begin{aligned} \hat{\boldsymbol{\mu}}
	=&\argmax_{\vmu}\lrp{\ell(\vmu)p(\vmu)}\\
	=& \underset{\boldsymbol{\mu}}{\arg \max }\left\{\left[-\frac{n}{2} \ln \left[(2 \pi)^{d}|\Sigma|\right]-\sum_{k=1}^{n} \frac{1}{2}\left(\mathbf{x}_{k}-\boldsymbol{\mu}\right)^{\T} \Sigma^{-1}\left(\mathbf{x}_{k}-\boldsymbol{\mu}\right)\right]\right.\\
	&\left. \cdot\left[\frac{1}{(2 \pi)^{d / 2}\left|\Sigma_{0}\right|^{1 / 2}} \exp \left[-\frac{1}{2}\left(\boldsymbol{\mu}-\mathbf{m}_{0}\right)^{\T} \Sigma_{0}^{-1}\left(\boldsymbol{\mu}-\mathbf{m}_{0}\right)\right]\right]\right\}
	\end{aligned}\]
	\item [(b)] 由均值和协方差的性质有
	\[\begin{aligned}\boldsymbol{\mu}^{\prime}&= \mathbb{E}\left[\mathbf{x}^{\prime}\right]=\mathbb{E}[A \mathbf{x}]=A \mathbb{E}[\mathbf{x}]=A \boldsymbol{\mu}\\
	\Sigma^{\prime} &=\mathbb{E}\left[\left(\mathbf{x}^{\prime}-\boldsymbol{\mu}^{\prime}\right)\left(\mathbf{x}^{\prime}-\boldsymbol{\mu}^{\prime}\right)^{\T}\right] \\
	&=\mathbb{E}\left[\left(A \mathbf{x}^{\prime}-A \boldsymbol{\mu}^{\prime}\right)\left(A \mathbf{x}^{\prime}-A \boldsymbol{\mu}^{\prime}\right)^{\T}\right] \\
	&=\mathbb{E}\left[A\left(\mathbf{x}^{\prime}-\boldsymbol{\mu}^{\prime}\right)\left(\mathbf{x}^{\prime}-\boldsymbol{\mu}^{\prime}\right)^{\T} A^{\T}\right] \\
	&=A \mathbb{E}\left[(\mathbf{x}^{\prime}-\boldsymbol{\mu}^{\prime})\left(\mathbf{x}^{\prime}-\boldsymbol{\mu}^{\prime}\right)^{\T}\right] A^{\T} \\
	&=A \Sigma A^{\T}
	\end{aligned}\]
	进而有$\vmu'$的对数似然函数
	\[\begin{aligned}
	\ell(\vmu')
	&=\ln \left(\prod_{k=1}^{n} p\left(\mathbf{x}_{k}^{\prime} | \boldsymbol{\mu}^{\prime}\right)\right) \\
	&=\sum_{k=1}^{n} \ln \left[p\left(A \mathbf{x}_{k} | A \boldsymbol{\mu}\right)\right] \\
	&=-\frac{n}{2} \ln \left[(2 \pi)^{d}\left|A \Sigma A^{\T}\right|\right]
	-\sum_{k=1}^{n} \frac{1}{2}\left((\mathbf{x}_k-\boldsymbol{\mu})^{\T} A^{\T}\right)
	\left(A \Sigma A^{\T}\right)^{-1}
	\left(A\left(\mathbf{x}_{k}-\boldsymbol{\mu}\right)\right) \\
	&=-\frac{n}{2} \ln \left[(2 \pi)^{d}\left|A \Sigma A^{\T}\right|\right]-\sum_{k=1}^{n} \frac{1}{2}\left(\mathbf{x}_{k}-\boldsymbol{\mu}\right)^{\T}\left(A^{\T}\left(A^{-1}\right)^{\T}\right) \Sigma^{-1}\left(A^{-1} A\right)\left(\mathbf{x}_{k}-\boldsymbol{\mu}\right) \\
	&=-\frac{n}{2} \ln \left[(2 \pi)^{d}\left|A \Sigma A^{\T}\right|\right]-\sum_{k=1}^{n} \frac{1}{2}\left(\mathbf{x}_{k}-\boldsymbol{\mu}\right)^{\T} \Sigma^{-1}\left(\mathbf{x}_{k}-\boldsymbol{\mu}\right)
	\end{aligned}\]
	类似地可以得到$\vmu'$的高斯密度
	\[\begin{aligned}
	p\left(\boldsymbol{\mu}^{\prime}\right)
	&=\frac{1}{(2 \pi)^{d / 2}\left|\Sigma_{0}^{\prime}\right|^{1 / 2}} \exp \left[-\frac{1}{2}\left(\boldsymbol{\mu}^{\prime}-\mathbf{m}_{0}'\right)^{\T} \Sigma_{0}^{\prime -1}\left(\boldsymbol{\mu}^{\prime}-\mathbf{m}_{0}'\right)\right] \\
	&=\frac{1}{(2 \pi)^{d / 2}\left|\Sigma_{0}^{\prime}\right|^{1 / 2}} \exp \left[-\frac{1}{2}\left(A \boldsymbol{\mu}-A \mathbf{m}_{0}\right)^{\T}\left(A \Sigma_{0} A^{\T}\right)^{-1}\left(A \boldsymbol{\mu}-A \mathbf{m}_{0}\right)\right] \\
	&=\frac{1}{(2 \pi)^{d / 2}\left|\Sigma_{0}^{\prime}\right|^{1 / 2}} \exp \left[-\frac{1}{2}\left(\boldsymbol{\mu}-\mathbf{m}_{0}\right)^{\T} A^{\T}\left(A^{-1}\right)^{\T} \Sigma_{0}^{-1} A^{-1} A\left(\boldsymbol{\mu}-\mathbf{m}_{0}\right)\right] \\
	&=\frac{1}{(2 \pi)^{d / 2}\left|\Sigma_{0}^{\prime}\right|^{1 / 2}} \exp \left[-\frac{1}{2}\left(\boldsymbol{\mu}-\mathbf{m}_{0}\right)^{\T} \Sigma_{0}^{-1}\left(\boldsymbol{\mu}-\mathbf{m}_{0}\right)\right]
	\end{aligned}\]
	新的MAP估计为
	\[\begin{aligned}
	\hat{\boldsymbol{\mu}}^{\prime}=& \underset{\boldsymbol{\mu}}{\arg \max }\left\{\left[-\frac{n}{2} \ln \left[(2 \pi)^{d}\left|A \Sigma A^{\T}\right|\right]
	-\sum_{k=1}^{n} \frac{1}{2}\left(\mathbf{x}_{k}-\boldsymbol{\mu}\right)^{\T} \Sigma^{-1}\left(\mathbf{x}_{k}-\boldsymbol{\mu}\right)\right]\right.\\
	&\left.\cdot\left[\frac{1}{(2 \pi)^{d / 2}\left|A \Sigma_0 A^{\T}\right|^{1 / 2}} \exp \left[-\frac{1}{2}\left(\boldsymbol{\mu}-\mathbf{m}_{0}\right)^{\T} \Sigma_{0}^{-1}\left(\boldsymbol{\mu}-\mathbf{m}_{0}\right)\right]\right]\right\}
	\end{aligned}\]
	与(a)比较知,$\hat{\vmu}$和$\hat{\vmu}'$的差异均在常数部分,因此MAP可以对变换后的$\vmu'$做出正确估计。
\end{itemize}
\end{answer}

\begin{question}[\textsection 3 Q38]
令$p_\vx(\vx\mid\omega_i),i=1,2$为任意的概率密度函数,均值为$\vmu_i$,协方差矩阵为$\Sigma_i$,其中并不要求$p_\vx(\vx\mid\omega_i)$必须为正态概率密度。
令$y=\vw^\T\vx$表示投影,并且设投影后的结果的概率密度函数为$p(y\mid\omega_i)$,其均值为$\mu_i$,方差为$\sigma^2_i$。
\begin{itemize}
	\item [(a)] 证明准则函数
	\[J_1(\vw)=\frac{(\mu_1-\mu_2)^2}{\sigma_1^2+\sigma_2^2}\]
	当
	\[\vw=(\Sigma_1+\Sigma_2)^{-1}(\vmu_1-\vmu_2)\]
	时取得最大值。
	\item [(b)] 如果$P(\omega_i)$为$\omega_i$的先验概率,证明
	\[J_2(\vw)=\frac{(\mu_1-\mu_2)^2}{P(\omega_1)\sigma_1^2+P(\omega_2)\sigma_2^2}\]
	当
	\[\vw=[P(\omega_1)\Sigma_1+P(\omega_2)\Sigma_2]^{-1}(\vmu_1-\vmu_2)\]
	时取得最大值。
	\item [(c)] 在(a)和(b)之间,哪个与公式(96)的联系更密切,请解释为什么。
\end{itemize}
\end{question}
\begin{answer}
\begin{itemize}
	\item [(a)] 对于$i=1,2$有
	\[\begin{aligned}
	\mu_{i} &=\frac{1}{n_{i}} \sum_{y \in \mathcal{Y}_{i}} y=\frac{1}{n_{i}} \sum_{\mathbf{x} \in \mathcal{D}_{i}} \mathbf{w}^{\T} \mathbf{x}=\mathbf{w}^{\T} \boldsymbol{\mu}_{i} \\
	\sigma_{i}^{2} &=\sum_{y \in \mathcal{Y}_{i}}\left(y-\mu_{i}\right)^{2}=\mathbf{w}^{\T}\left[\sum_{\mathbf{x} \in \mathcal{D}_{i}}\left(\mathbf{x}-\boldsymbol{\mu}_{i}\right)\left(\mathbf{x}-\boldsymbol{\mu}_{i}\right)^{\T}\right] \mathbf{w}=\vw^\T\Sigma_i\vw \\
	\Sigma_{i} &=\sum_{\mathbf{x} \in \mathcal{D}_{i}}\left(\mathbf{x}-\boldsymbol{\mu}_{i}\right)\left(\mathbf{x}-\boldsymbol{\mu}_{i}\right)^{\T}
	\end{aligned}\]
	有总类内散度矩阵$S_W$及总类间散度矩阵$S_B$
	\[\begin{aligned}
	S_W &=\Sigma_{1}+\Sigma_{2} \\
	S_B &=\left(\boldsymbol{\mu}_{1}-\boldsymbol{\mu}_{2}\right)\left(\boldsymbol{\mu}_{1}-\boldsymbol{\mu}_{2}\right)^{\T}
	\end{aligned}\]
	进而
	\[J_{1}(\mathbf{w})=\frac{(\mu_1-\mu_2)^2}{\sigma_1^2+\sigma_2^2}=\frac{\mathbf{w}^{\T} S_B \mathbf{w}}{\mathbf{w}^{\T} S_W \mathbf{w}}\]
	即广义瑞利商。
	由课本公式(106),可以得到$J_1(\vw)$在$\mathbf{w}= S_W^{-1}\left(\boldsymbol{\mu}_{1}-\boldsymbol{\mu}_{2}\right)$得到最大值,即在
	\[\vw=(\Sigma_1+\Sigma_2)^{-1}(\vmu_1-\vmu_2)\]
	时取得最大值。
	\item [(b)] 同(a)理,用广义瑞利商表示有
	\[J_2(\vw)=\frac{(\mu_1-\mu_2)^2}{P(\omega_1)\sigma_1^2+P(\omega_2)\sigma_2^2}=
	\frac{\vw^\T S_B\vw}{\vw^\T S_W' \vw}\]
	其中$S_W'=P(\omega_1)\sigma_1^2+P(\omega_2)\sigma_2^2$。
	同理有$J_2(\vw)$在
	\[\vw=S_W'^{-1}\left(\boldsymbol{\mu}_{1}-\boldsymbol{\mu}_{2}\right)=
	[P(\omega_1)\Sigma_1+P(\omega_2)\Sigma_2]^{-1}(\vmu_1-\vmu_2)\]
	时取得最大值。
	\item [(c)] 在公式(96)中,令$\tilde{m}_i=\mu_i$和$\tilde{s}_i^2=\sigma_i^2$,可以得到(a)中的式子,即(96)与(a)联系更密切。
\end{itemize}
\end{answer}

\end{document}