% !TEX root = main.tex

\subsection{各组件实现}
\qquad 各组件实现详情见第\ref{sec:appendix}章节,均有详细注释。在这里仅仅提一些需要注意的点。
\begin{enumerate}
	\item 指令存储器和数据存储器均采用小端(little endian)存储,并且位宽为8位,故生成的指令文件需要进行预处理,最先读入的应该是指令的低8位。这里用\verb'Python'进行MIPS程序的编译并生成指令二进制代码文件,读入时直接将其初始化到指令存储器中。
	\item 程序计数器(PC)需要初始化置零(\verb'0x000000'),遇到reset信号时同样置零,而且只有在PCWrite=1时才更新PC。(实际操作时是置为$-4=$\verb'0xFFFFFFFC',为使第一条指令能够正常执行完全部状态)
	\item 多路选择器(MUX)需要实现一个32位和一个5位的,32位可以复用(创建多个实例);且需要构造2输入、3输入、4输入三种不同的MUX。
	\item ALU的功能见表\ref{tab:alu_op}。
    \item 控制单元的设计见第\ref{sub:control_unit}节。
\end{enumerate}
% Table generated by Excel2LaTeX from sheet 'ALUOp'
\begin{table}[H]
  \centering\xiaowu
  \caption{ALU功能码}
    \begin{tabular}{|c|l|l|}
    \hline
    ALUOp[2:0] & \multicolumn{1}{c|}{功能} & \multicolumn{1}{c|}{描述} \\
    \hline
    000   & $Y=A+B$ & 加 \\
    \hline
    001   & $Y=A-B$ & 减 \\
    \hline
    010   & $Y=B<<A$ & 左移 \\
    \hline
    011   & $Y=A\lor B$ & 逻辑或 \\
    \hline
    100   & $Y=A\land B$ & 逻辑与 \\
    \hline
    101   & $Y=(A<B)\;?\;1\;:\;0$ & 比较无符号数 \\
    \hline
    110   & \multicolumn{1}{p{7cm}|}{$\begin{aligned}
    Y&=(((A<B) \&\& (A[31] == B[31] ))\\
    &||\quad( ( A[31] ==1 \&\& B[31] == 0)))\\
    &?\;1\;:\;0
    \end{aligned}$} & 比较有符号数 \\
    \hline
    111   & $Y=A\oplus B$ & 逻辑异或 \\
    \hline
    \end{tabular}%
  \label{tab:alu_op}%
\end{table}%