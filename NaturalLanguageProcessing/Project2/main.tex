\documentclass[logo,reportComp]{thesis}
\usepackage[python,pseudo]{mypackage}

\title{自然语言处理}
\subtitle{期末大项目:基于深度学习的中英机器翻译}
\school{数据科学与计算机学院}
\author{陈鸿峥}
\classname{17大数据与人工智能}
\stunum{17341015}
\headercontext{自然语言处理作业}

\let\emph\relax % there's no \RedeclareTextFontCommand
\DeclareTextFontCommand{\emph}{\kaiti\em}

\begin{document}

\maketitle
\tableofcontents

\newpage

\section{预处理}

% \begin{thebibliography}{99}

% \end{thebibliography}

\end{document}
% 要求:
% 1. 描述清楚核心代码逻辑和tensor维度
% 2. 描述清楚代码的运行环境和软件版本
% 3. 独立完成,不得抄袭!不得抄袭!不得抄袭

% 参考教程:
% pytorch.org/tutorias/intermediate/seq2seq_translation_tutorial.html
% tensorflow.google.cn/tutorials/text/nmt_with_attention

% 相关论文:
% Effective Approaches to Attention-based Neural Machine Translation, Luong et al., EMNLP 2015
% Neural Machine Translation by Jointly Learning to Align and Translate, Bahdanau et al., ICLR 2015
% Bleu: a Method for Automatic Evaluation for Machine Translation, Papineni et al., ACL 2002