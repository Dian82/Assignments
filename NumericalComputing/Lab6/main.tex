\documentclass[reportComp]{thesis}
\usepackage[cpp,linenum]{mypackage}

\title{数值计算方法实验报告}
\subtitle{实验五:常微分方程数值解}
\school{数据科学与计算机学院}
\author{陈鸿峥}
\classname{17大数据与人工智能}
\stunum{17341015}
\headercontext{数值计算方法实验报告}

\begin{document}

\maketitle

\section{实验题目}
用标准四阶龙格-库塔方法,对$x\geq 0$时的标准正态分布函数
\[\phi(x)=\frac{1}{\sqrt{2\pi}}\intabu{0}{x}{\ee^{-\frac{t^2}{2}}}{t}+\frac{1}{2},\;0\leq x<\infty\]
产生一张在$[0,5]$之间$80$个等距节点处的函数表。

\section{实验目的}
理解常微分方程的数值解法,会利用龙格-库塔方法求解微分方程。

\section{实验原理与内容}
% 若有推导的式子,写在这里
% 重要代码和截图
原式等价于下面的微分方程
\[\begin{cases}
\phi'(x)=\frac{1}{\sqrt{2\pi}}\ee^{-\frac{x^2}{2}}\\
\phi(0)=\frac{1}{2}
\end{cases}\]
进而可用四阶龙格-库塔公式
\[\begin{cases}
k_1=hf(x_n,y_n)\\
k_2=hf\lrp{x_n+\frac{1}{2}h,y_n+\frac{1}{2}k_1}\\
k_3=hf\lrp{x_n+\frac{1}{2}h,y_n+\frac{1}{2}k_2}\\
k_4=hf\lrp{x_n+h,y_n+k_3}\\
y_{n+1}=y_n+\frac{1}{6}(k_1+2k_2+2k_3+k_4)
\end{cases}\]
其中,
\[f(x,y)=\frac{1}{\sqrt{2\pi}}\ee^{-\frac{x^2}{2}}\]

下面为本次实验的Mathematica源代码,完整文件已在附件中\verb'Runge-Kutta.nb'。
\begin{lstlisting}[language=mathematica]
f[x_] := N[1/Sqrt[2 \[Pi]] E^(-x^2/2)];
a = 0; b = 5; n = 80;
h = (b - a)/n;
y = Table[0, {i, 1, 81}];
y[[1]] = 1/2;
For[i = 1, i <= n, i++,
 xn = a + (i - 1)*h;
 k1 = h*f[xn];
 k2 = h*f[xn + 1/2*h];
 k3 = h*f[xn + 1/2*h];
 k4 = h*f[xn + h];
 y[[i + 1]] = y[[i]] + 1/6*(k1 + 2 k2 + 2 k3 + k4)]
Print[y]
\end{lstlisting}

\section{实验结果与分析}
% 对运行结果说明(图像截图,数据列成表)并分析
结果如下表所示,注意包含了首尾节点,故一共打印了$81$个值。
\begin{table}[H]
\begin{tabular}{cccccccc}\hline
0.5 & 0.524918 & 0.549738 & 0.574366 & 0.598706 & 0.62267 & 0.64617 & 0.669126 \\\hline
0.691462 & 0.713112 & 0.734014 & 0.754116 & 0.773373 & 0.791748 & 0.809213 & 0.825749 \\\hline
0.841345 & 0.855996 & 0.869705 & 0.882485 & 0.89435 & 0.905324 & 0.915434 & 0.924712 \\\hline
0.933193 & 0.940915 & 0.947919 & 0.954246 & 0.959941 & 0.965046 & 0.969604 & 0.973658 \\\hline
0.97725 & 0.98042 & 0.983207 & 0.985647 & 0.987776 & 0.989625 & 0.991226 & 0.992605 \\\hline
0.99379 & 0.994804 & 0.995668 & 0.996401 & 0.99702 & 0.997542 & 0.99798 & 0.998346 \\\hline
0.99865 & 0.998903 & 0.999111 & 0.999282 & 0.999423 & 0.999538 & 0.999631 & 0.999706 \\\hline
0.999767 & 0.999816 & 0.999856 & 0.999887 & 0.999912 & 0.999931 & 0.999947 & 0.999959 \\\hline
0.999968 & 0.999976 & 0.999981 & 0.999986 & 0.999989 & 0.999992 & 0.999994 & 0.999995 \\\hline
0.999997 & 0.999997 & 0.999998 & 0.999999 & 0.999999 & 0.999999 & 0.999999 & 1.\\\hline
1. & & & & & & & \\\hline
\end{tabular}
\end{table}

可以见得龙格-库塔方法得出的正态分布表与我们以前学习概率论时记忆的正态分布表一致,精度非常高。

\section{实验总结和心得}
本次实验明白了数值求解常微分方程的原理,并且实现了龙格-库塔方法,收获良多。

\end{document}
% 1024180018@qq.com
% shuzhijisuan2017@163.com

% ftp://172.18.216.222
% shuzhi2019
% 学号_姓名_weekX_vY

% 上机作业要求
% 建议使用C++/Matlab编程,注:不允许使用内置函数完成主要功能
% 主题/文件名:班级+姓名(小组)+学号+第几次作业
% 实验报告(运行结果)、源代码