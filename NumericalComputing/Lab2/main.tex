\documentclass[reportComp]{thesis}
\usepackage[cpp,linenum]{mypackage}

\title{数值计算方法实验报告}
\subtitle{实验二:追赶法解线性方程组}
\school{数据科学与计算机学院}
\author{陈鸿峥}
\classname{17大数据与人工智能}
\stunum{17341015}
\headercontext{数值计算方法实验报告}

\begin{document}

\maketitle

\section{实验题目}
写出用追赶法求解下述线性方程组的程序,其中$n=101$
\[\bmat{12 & 1 & 0 & \cdots & 0\\1 & 12 & 1 & \cdots & 0\\0 & 1 & 12 & \ddots & \vdots\\\vdots & \vdots & \ddots & \ddots & 1\\0 & 0 & \cdots & 1 & 12}\bmat{x_1\\x_2\\x_3\\\vdots\\x_n}=\bmat{11\\10\\10\\\vdots\\11}\]

\section{实验目的}
理解追赶法解三对角方程的原理,并实现相关程序。

\section{实验原理与内容}
% 若有推导的式子,写在这里
% 重要代码和截图
由课本2.2.3节,给出一般的三对角方程解法,设$n$阶方程组$A\vx=\vd$,其中
\[A=\bmat{b_1 & c_1 & & & \\a_2 & b_2 & c_2 & & \\ & \ddots & \ddots & \ddots & \\ & & a_{n-1} & b_{n-1} & c_{n-1}\\ & & & a_n & b_n},\;\vd=\bmat{d_1\\d_2\\\vdots\\d_{n-1}\\d_n}\]
对矩阵$A$做克洛脱分解,有
\[L=\bmat{l_1 & & & &\\v_2 & l_2 & & & \\ & \ddots & \ddots & & \\& & v_{n-1} & l_{n-1} &\\ & & & v_n & l_n},\;U=\bmat{1 & u_1 & & &\\ & 1 & u_2 & & \\ & & \ddots & \ddots & \\ & & & 1 & u_{n=1}\\ & & & & 1}\]
进而$A=LU$,做两次回代即可
\[\begin{cases}
L\vy=\vd\\
U\vx=\vy
\end{cases}\]
可得下面程序中的公式,其中\verb'Forward'为追的过程,\verb'Backward'为赶的过程。

下面为本次实验的Mathematica源代码,完整文件已在附件中\verb'TriDiagSolver.nb'。
\begin{lstlisting}[language=mathematica]
AllMat[j_, n_] := Table[j, {i, 1, n}];
ResMat[n_] := Table[If[i == 1 || i == n, 11, 10], {i, 1, n}];
ZeroMat[n_] := Table[0, {i, 1, n}];
Num = 101;
(*Initialization*)
a = AllMat[1, Num]; b = AllMat[12, Num]; c = AllMat[1, Num];
d = ResMat[Num];
l = ZeroMat[Num]; y = ZeroMat[Num]; u = ZeroMat[Num]; x = ZeroMat[Num];
(*TriDiagSolver[n_]*)
TriDiagSolverForward[n_] := For[i = 1, i <= n, i++,
   If[i == 1, l[[i]] = b[[i]],
    If[i == n, l[[i]] = b[[n]] - a[[n - 1]]*u[[n - 1]],
     l[[i]] = b[[i]] - a[[i - 1]]*u[[i - 1]]]];
   If[i == 1, y[[1]] = d[[1]]/l[[1]],
    If[i == n, y[[n]] = (d[[n]] - y[[n - 1]]*a[[n - 1]])/l[[n]],
     y[[i]] = (d[[i]] - y[[i - 1]]*a[[i - 1]])/l[[i]]]];
   If[i == 1, u[[1]] = c[[1]]/l[[1]],
    If[i < n, u[[i]] = c[[i]]/l[[i]],
     0]]];
TriDiagSolverBackward[n_] := For[i = n, i > 0, i--,
   If[i == n, x[[n]] = y[[n]],
    x[[i]] = y[[i]] - u[[i]]*x[[i + 1]]]];
TriDiagSolverForward[Num];
TriDiagSolverBackward[Num];
N[x]
\end{lstlisting}

\section{实验结果与分析}
% 对运行结果说明(图像截图,数据列成表)并分析
下面是$\vx$的结果,一共$101$个元素,为了编排分成$11$行。
{\xiaowu
\begin{center}
\begin{tabular}{cccccccccc}\hline\hline
0.858149 & 0.702213 & 0.715299 & 0.714201 & 0.714293 & 0.714285 & 0.714286 & 0.714286 & 0.714286 & 0.714286 \\
0.714286 & 0.714286 & 0.714286 & 0.714286 & 0.714286 & 0.714286 & 0.714286 & 0.714286 & 0.714286 & 0.714286 \\
0.714286 & 0.714286 & 0.714286 & 0.714286 & 0.714286 & 0.714286 & 0.714286 & 0.714286 & 0.714286 & 0.714286 \\
0.714286 & 0.714286 & 0.714286 & 0.714286 & 0.714286 & 0.714286 & 0.714286 & 0.714286 & 0.714286 & 0.714286 \\
0.714286 & 0.714286 & 0.714286 & 0.714286 & 0.714286 & 0.714286 & 0.714286 & 0.714286 & 0.714286 & 0.714286 \\
0.714286 & 0.714286 & 0.714286 & 0.714286 & 0.714286 & 0.714286 & 0.714286 & 0.714286 & 0.714286 & 0.714286 \\
0.714286 & 0.714286 & 0.714286 & 0.714286 & 0.714286 & 0.714286 & 0.714286 & 0.714286 & 0.714286 & 0.714286 \\
0.714286 & 0.714286 & 0.714286 & 0.714286 & 0.714286 & 0.714286 & 0.714286 & 0.714286 & 0.714286 & 0.714286 \\
0.714286 & 0.714286 & 0.714286 & 0.714286 & 0.714286 & 0.714286 & 0.714286 & 0.714286 & 0.714286 & 0.714286 \\
0.714286 & 0.714286 & 0.714286 & 0.714286 & 0.714286 & 0.714285 & 0.714293 & 0.714201 & 0.715299 & 0.702213 \\
0.858149\\\hline
\end{tabular}
\end{center}
}

通过将结果回代验证,结果是正确的,且精度非常高。

本实验说明用追赶法解三对角方程对于计算机来说非常的简单,而且可以简便快捷地得到很高精度的结果。

\section{实验总结和心得}
本次实验较为简单,只需将三对角方程的算法在计算机上实现一遍即可,而且由于是简单的循环赋值,用其他高级语言如C++/Python等也可轻松实现。
而轻松实现的前提是有深厚的数学基础,由此看来数值计算的作用非常强大,既能告诉你如何解出高精度的正确结果,又能通过优化算法尽可能快地解出答案。

\end{document}
% 1024180018@qq.com
% shuzhijisuan2017@163.com

% ftp://172.18.216.170
% shuzhi2019
% 学号_姓名_weekX_vY

% 上机作业要求
% 建议使用C++/Matlab编程,注:不允许使用内置函数完成主要功能
% 主题/文件名:班级+姓名(小组)+学号+第几次作业
% 实验报告(运行结果)、源代码