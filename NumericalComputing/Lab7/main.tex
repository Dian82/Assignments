\documentclass[reportComp]{thesis}
\usepackage[cpp,linenum]{mypackage}

\title{数值计算方法实验报告}
\subtitle{实验六:常微分方程数值解}
\school{数据科学与计算机学院}
\author{陈鸿峥}
\classname{17大数据与人工智能}
\stunum{17341015}
\headercontext{数值计算方法实验报告}

\begin{document}

\maketitle

\section{实验题目}
尝试用不同方法求解下面的初值问题
\[\bmat{u'\\v'}=\bmat{32 & 66\\-66&-133}\bmat{u\\v}+\bmat{\frac{2}{3}x+\frac{2}{3}\\-\frac{1}{3}x+\frac{1}{3}},\,x\in[0,0.5]\]
初值条件为
\[\bmat{u(0)\\v(0)}=\bmat{\frac{1}{3}\\\frac{1}{3}}\]
比较各种方法的计算结果和计算时间。(该问题的精确解为$u=\frac{2}{3}x+\frac{2}{3}\ee^{-x}-\frac{1}{3}\ee^{-100x},v=-\frac{1}{3}x-\frac{1}{3}\ee^{-x}+\frac{2}{3}\ee^{-100x}$)。

\section{实验目的}
理解常微分方程的数值解法,会利用不同方法求解微分方程。

\section{实验原理与内容}
% 若有推导的式子,写在这里
% 重要代码和截图
本次实验主要对欧拉公式、改进欧拉公式和龙格-库塔四阶公式进行探究。
记
\[A=\bmat{32 & 66\\-66&-133},\;\vy_0=\bmat{u(0)\\v(0)}=\bmat{\frac{1}{3}\\\frac{1}{3}},\;\vc(x)=\bmat{\frac{2}{3}x+\frac{2}{3}\\-\frac{1}{3}x+\frac{1}{3}}\]
则
\[f(x,y)=A\vy+\vc(x)\]
有欧拉公式
\[\vy_{n+1}=\vy_n+hf(x_n,\vy_n)\]
改进欧拉公式
\[\begin{cases}
\vy_p=\vy_n+hf(x_n,\vy_n)\\
\vy_c=\vy_n+hf(x_{n+1},\vy_p)\\
\vy_{n+1}=\frac{1}{2}(\vy_p+\vy_c)
\end{cases}\]
标准四阶四段龙格-库塔公式
\[\begin{cases}
k_1=hf(x_n,\vy_n)\\
k_2=hf\lrp{x_n+\frac{1}{2}h,\vy_n}\\
k_3=hf\lrp{x_n+\frac{1}{2}h,\vy_n+\frac{1}{2}k_1}\\
k_4=hf\lrp{x_n+h,\vy_n+k_3}\\
\vy_{n+1}=\vy_n+\frac{1}{6}(k_1+2k_2+2k_3+k_4)
\end{cases}\]

下面为本次实验的Mathematica源代码,完整文件已在附件中\verb'ODE.nb'。
\begin{lstlisting}[language=mathematica]
u[x_] := 2/3 x + 2/3 E^-x - 1/3 E^(-100 x);
v[x_] := -1/3 x - 1/3 E^-x + 2/3 E^(-100 x);
a = 0; b = 0.5;
NList = {5, 50, 500, 5000, 50000};
x = Table[a + i*h, {i, 0, n}];
yn = {u[b], v[b]};
A = {{32, 66}, {-66, -133}};
c[x_] := {2/3 x + 2/3, -1/3 x + 1/3};
f[x_, y_] := A.y + c[x];
Do[h = (b - a)/n;
 (*Euler*)
 y = Table[Table[0, {i, 1, 2}], {j, 1, n + 1}];
 y[[1]] = {1/3, 1/3};
 time = TimeUsed[];
 For[i = 1, i <= n, i++,
   y[[i + 1]] = y[[i]] + h*f[x[[i]], y[[i]]]]
  Print["MaxIter: ", n, "\t Euler err: ", Norm[y[[n + 1]] - yn, 1], 
   "\t Time: ", TimeUsed[] - time];
 (*Improved Euler*)
 y = Table[Table[0, {i, 1, 2}], {j, 1, n + 1}];
 y[[1]] = {1/3, 1/3};
 time = TimeUsed[];
 For[i = 1, i <= n, i++,
   yp = y[[i]] + h*f[x[[i]], y[[i]]];
   yc = y[[i]] + h*f[x[[i + 1]], yp];
   y[[i + 1]] = 1/2*(yp + yc)]
  Print["MaxIter: ", n, "\t Improved Euler err: ", 
   Norm[y[[n + 1]] - yn, 1], "\t Time: ", TimeUsed[] - time];
 (*Runge-Kutta*)
 y = Table[Table[0, {i, 1, 2}], {j, 1, n + 1}];
 y[[1]] = {1/3, 1/3};
 time = TimeUsed[];
 For[i = 1, i <= n, i++,
   k1 = h*f[x[[i]], y[[i]]];
   k2 = h*f[x[[i]] + 1/2*h, y[[i]] + 1/2*k1];
   k3 = h*f[x[[i]] + 1/2*h, y[[i]] + 1/2*k2];
   k4 = h*f[x[[i]] + h, y[[i]] + k3];
   y[[i + 1]] = y[[i]] + 1/6*(k1 + 2 k2 + 2 k3 + k4)]
  Print["MaxIter: ", n, "\t Runge-Kutta err: ", 
   Norm[y[[n + 1]] - yn, 1], "\t Time: ", TimeUsed[] - time], {n, 
  NList}]
\end{lstlisting}

\section{实验结果与分析}
% 对运行结果说明(图像截图,数据列成表)并分析
\begin{center}
\begin{tabular}{|c|c|c|c|}\hline
\textbf{方法} & \textbf{最大迭代轮数} & \textbf{绝对误差} & \textbf{运行时间(s)}\\\hline\hline
欧拉 & 5	& 58261.8	 & 0.\\\hline
改进欧拉 & 5	& 1.14311$\cdot 10^8$	 & 0.\\\hline
龙格-库塔 & 5	& 2.0589$\cdot 10^{12}$	 & 0.\\\hline\hline
欧拉 & 50	& 0.14357	 & 0.\\\hline
改进欧拉 & 50	& 0.142552	 & 0.\\\hline
龙格-库塔 & 50	& 0.144524	 & 0.\\\hline\hline
欧拉 & 500	& 0.143614	 & 0.016\\\hline
改进欧拉 & 500	& 0.143514	 & 0.\\\hline
龙格-库塔 & 500	& 0.143709	 & 0.031\\\hline\hline
欧拉 & 5000	& 0.153111	 & 0.078\\\hline
改进欧拉 & 5000	& 0.153102	 & 0.281\\\hline
龙格-库塔 & 5000	& 0.15312	 & 0.359\\\hline\hline
欧拉 & 50000	& 0.248979	 & 0.547\\\hline
改进欧拉 & 50000	& 0.24898	 & 1.25\\\hline
龙格-库塔 & 50000	& 0.24898	 & 2.187\\\hline
\end{tabular}
\end{center}

从上表可以得出以下结论:
\begin{itemize}
	\item 不同方法都随着迭代轮数的增加而精度增加
	\item 当迭代轮数比较少时,欧拉公式的绝对误差较小
	\item 随着迭代轮数的增加,改进欧拉公式和四阶龙格-库塔公式都取得了较低的绝对误差
	\item 欧拉公式的速度最快,龙格-库塔的速度最慢,某种程度上速度与精度呈负相关关系
\end{itemize}

\section{实验总结和心得}
本次实验明白了数值求解常微分方程的原理,并且实现了多种数值方法,收获良多。

\end{document}
% 1024180018@qq.com
% shuzhijisuan2017@163.com

% ftp://172.18.216.222
% shuzhi2019
% 学号_姓名_weekX_vY

% 上机作业要求
% 建议使用C++/Matlab编程,注:不允许使用内置函数完成主要功能
% 主题/文件名:班级+姓名(小组)+学号+第几次作业
% 实验报告(运行结果)、源代码