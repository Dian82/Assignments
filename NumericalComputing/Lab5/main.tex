\documentclass[reportComp]{thesis}
\usepackage[cpp,linenum]{mypackage}

\title{数值计算方法实验报告}
\subtitle{实验五:数值积分}
\school{数据科学与计算机学院}
\author{陈鸿峥}
\classname{17大数据与人工智能}
\stunum{17341015}
\headercontext{数值计算方法实验报告}

\begin{document}

\maketitle

\section{实验题目}
在每一个离散节点$a=x_0<x_1<\cdots<x_n<x_{n+1}$上分别采用两点公式(5.50)和公式(5.51)求离散微分初值问题
\[\begin{cases}
y'=f(x,y) & x>a\\
y(a)=y_0
\end{cases}\]
中的导数$y'$,从而推导出求解微分方程初值问题的两个不同的计算格式---显格式和隐格式,分别说明其计算步骤。

\section{实验目的}
理解并运用数值微分方法。

\section{实验原理与内容}
% 若有推导的式子,写在这里
% 重要代码和截图
由两点公式(5.50)有
\[y'(x_i)=f(x_i,y_i)\approx\frac{y_{i+1}-y_i}{x_{i+1}-x_i}\]
进而
\[y_{i+1}\approx y_i+(x_{i+1}-x_i)f(x_i,y_i),i=0,1,2,\ldots,N-1\]
上式右侧函数$f$内的变量下标均为$i$,为显格式。

由两点公式(5.51)有
\[y'(x_{i+1})=f(x_{i+1},y_{i+1})\approx\frac{y_{i+1}-y_i}{x_{i+1}-x_i}\]
进而
\[y_{i+1}\approx y_i+(x_{i+1}-x_i)f(x_{i+1},y_{i+1}),i=0,1,2,\ldots,N-1\]
上式右侧函数$f$内的变量下标为$i+1$,为隐格式。

\end{document}
% 1024180018@qq.com
% shuzhijisuan2017@163.com

% ftp://172.18.216.222
% shuzhi2019
% 学号_姓名_weekX_vY

% 上机作业要求
% 建议使用C++/Matlab编程,注:不允许使用内置函数完成主要功能
% 主题/文件名:班级+姓名(小组)+学号+第几次作业
% 实验报告(运行结果)、源代码