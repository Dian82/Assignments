\documentclass[logo,reportComp]{thesis}
\usepackage[python,pseudo,linenum]{mypackage}
\usepackage{relsize}

\setcounter{secnumdepth}{4}
\titleformat{\paragraph}{\bfseries}{\alph{paragraph})~~}{0em}{}
\titlespacing*{\paragraph}{0pt}{0pt}{0pt}[0pt]

\title{人工神经网络}
\subtitle{Lab 3:全连接神经网络(FCN)}
\school{数据科学与计算机学院}
\author{陈鸿峥}
\classname{17大数据与人工智能}
\stunum{17341015}
\headercontext{人工神经网络作业}

\let\emph\relax % there's no \RedeclareTextFontCommand
\DeclareTextFontCommand{\emph}{\kaiti\em}
\AtBeginEnvironment{quote}{\kaiti\small}

\begin{document}

\maketitle
\tableofcontents

\newpage

\section{项目概览}

\end{document}
% 请提交一份简短的实验报告,说明神经网络的实现过程以及模型在数据集上的表现。代码应有适量注释,并与报告一起提交。
% 说明:
% (1) 需要设计部分网络的结构,比如两层隐含层的神经元数,激活函数等;
% (2) 全连接层的参数初始化无需自己实现, 可直接调用函数;
% (3) 对类的设计没有具体要求, 在代码注释或报告中简要说明即可;