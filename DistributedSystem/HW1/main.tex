\documentclass[logo,reportComp]{thesis}
\usepackage[cpp,pseudo]{mypackage}

\title{分布式系统作业一}
\subtitle{}
\school{数据科学与计算机学院}
\author{陈鸿峥}
\classname{17大数据与人工智能}
\stunum{17341015}
\headercontext{分布式系统作业}

\begin{document}

\maketitle

\begin{question}
在分布式系统中,为什么有时难以隐藏故障的发生以及故障恢复过程?
\end{question}
\begin{answer}
一方面我们很难知道一个分布式系统究竟是跑得慢(因为负载不均、网络不通畅等原因),还是真实出现了故障,这是需要具体情况具体分析的。

另一方面以现在分布式系统的自动化程度,是很难将这些故障的出现和恢复全部交由系统来做的(就像现在的编译器也没有办法做到自动帮我们修复bug一样),因此总要有人类介入(通过warning、error等方式告诉用户)。
\end{answer}

\begin{question}
给出一些体现分布式系统可扩展性的案例,并解释?
\end{question}
\begin{answer}
可扩展性包括规模可扩展、地理可扩展和管理可扩展。
\begin{itemize}
	\item 规模可扩展:近几年亚马逊\cite{bib:aws}、微软\cite{bib:azure}等云服务都至少翻了个倍,用户也不断在激增。
	\item 地理可扩展:这对于现在的云服务厂商来说也十分普遍,如微软将新的数据中心部署在苏格兰奥克尼群岛海底,Facebook将服务器部署在北极圈内,腾讯将自己的数据中心扩展到贵州\cite{bib:geo}等等。
	\item 管理可扩展:比如迅雷的P2P(BitTorrent)\cite{bib:xunlei}系统
\end{itemize}
\end{answer}

\begin{question}
列举应用程序之间的通信方式,并简单解释?
\end{question}
\begin{answer}
主要是下列两种模式:
\begin{itemize}
	\item Client-Server模式:即中心化的分布式系统,一个Client与多个Server进行通信,可以实现一对一、一对多或多对一通信。
	\item P2P模式:即去中心化的分布式系统,所有服务器都是对等实体,互相之间可以通信。
\end{itemize}
但这个问题问得有点大,如果涉及到具体的通信协议,传统的采用TCP/IP的方式进行传输,而现在很多分布式系统都开始采用RDMA\cite{bib:rdma}的方式。
而且近年来的通信架构也有很大程度的发展,无论是PCIe\cite{bib:pcie}、InfiniBand\cite{bib:infiniband},还是现在英伟达推的NvLink\cite{bib:nvlink}都将分布式系统的通信速度推向新的高峰。
\end{answer}

\begin{thebibliography}{99}
\bibitem{bib:aws} 亚马逊云AWS,\url{https://aws.amazon.com/cn/?nc2=h_lg}
\bibitem{bib:azure} 微软云Azure,\url{https://azure.microsoft.com/zh-cn/}
\bibitem{bib:geo} 科技巨头们的数据中心,\url{https://baijiahao.baidu.com/s?id=1609550999919190110&wfr=spider&for=pc}
\bibitem{bib:xunlei} 迅雷P2P,\url{https://www.xunlei.com/}
\bibitem{bib:rdma} 远程直接内存访问(RDMA),\url{https://en.wikipedia.org/wiki/Remote_direct_memory_access}
\bibitem{bib:pcie} PCIe,\url{https://en.wikipedia.org/wiki/PCI_Express}
\bibitem{bib:infiniband} InfiniBand,\url{https://en.wikipedia.org/wiki/InfiniBand}
\bibitem{bib:nvlink} NvLink,\url{https://www.nvidia.com/en-us/data-center/nvlink/}
\end{thebibliography}

\end{document}