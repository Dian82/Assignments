\documentclass[a4paper, 11pt]{article}
\usepackage[UTF8]{ctex}
\usepackage[cpp,table]{mypackage}
\usepackage{cancel,ulem}
\usepackage{amsmath}
\usepackage{graphicx}
\usepackage{geometry}
\usepackage{listings}
\geometry{scale=0.8}
% \linespread{1.5}
\usepackage[colorlinks,linkcolor=red]{hyperref}

\title{
\normalfont \normalsize
\textsc{School of Data and Computer Science, Sun Yat-sen University} \\ [25pt] %textsc small capital letters
\rule{\textwidth}{0.5pt} \\[0.4cm] % Thin top horizontal rule
\huge  T02 CSP and KRR\\ % The assignment title
\rule{\textwidth}{2pt} \\[0.5cm] % Thick bottom horizontal rule
\author{17341015 陈鸿峥}
\date{\normalsize\today}
}

\begin{document}
\maketitle
\tableofcontents
\newpage

\begin{flushleft}
\textemph{\Large Score: 13/15 + 35/35 + 13/15 + 35/35 = 96/100}
\end{flushleft}

\section{Q1 - CSP}
\begin{question}\normalfont
    Give precise formulations for each of the following as CSPs: Provide the variables, domains and constraints.
    \begin{itemize}
    \item [(a)] Crossword puzzle: shown below. We want to find six three-letter words: three words read across (A1, A2, and A3) and three words read down (D1, D2, and D3). Each word must be chosen from the list of forty possible words.
    \begin{center}
        {\scriptsize
        \begin{tabular}{|p{1cm}|p{1cm}|p{1cm}|}\hline
            A1, D1 & D2 & D3\\[1cm]\hline
            A2 & & \\[1cm]\hline
            A3 & & \\[1cm]\hline
        \end{tabular}}
        {\small
        \begin{tabular}{l}
            Word list:\\
            add, ado, age, ago, aid,\\
            ail, aim, air, and, any,\\
            ape, apt, arc, are, ark,\\
            arm, art, ash, ask, auk,\\
            awe, awl, aye, bad, bag,\\
            ban, bat, bee, boa, ear,\\
            eel, eft, far, fat, fit,\\
            lee, oaf, rat, tar, tie.
        \end{tabular}}
    \end{center}
    \item [(b)] Independent set: Given a graph and a number $k$, find an independent set of size $k$, that is, a set of $k$ vertices, no two of which are adjacent.
    \item [(c)] Crypto-arithmetic puzzle: $\text{SEND} + \text{MORE} = \text{MONEY}$. We want to replace each letter by a different digit so that the equation is correct.
    \end{itemize}
\end{question}
\begin{answer}
\begin{itemize}
\item [(a)] 设第$(i,j)$个格可以填入的字母为\textbf{变量}$V_{ij}$,其中$i$代表行号,$j$代表列号,$i,j\in\{1,2,3\}$。
设单词表中每个单词的第$k$个字母构成集合$W_k(k=1,2,3)$,有
\[\begin{aligned}
    W_1 &= \{a,b,e,f,l,o,r,t\}\\
    W_2 &= \{a,d,e,f,g,i,n,o,p,r,s,u,w,y\}\\
    W_3 &= \{a,c,d,e,f,g,h,k,l,m,n,o,r,t,y\}
\end{aligned}\]
在不引起歧义的情况下,这里忽略字符的引号。
有以下\textbf{论域}:
\[\begin{aligned}
\dom[V_{11}] &= W_1\cap W_1 = \{a,b,e,f,l,o,r,t\}\\
\dom[V_{12}] &= W_1\cap W_2 = \{a,e,f,o,r\}\\
\dom[V_{13}] &= W_1\cap W_3 = \{a,e,f,l,o,r,t\}\\
\dom[V_{21}] &= W_2\cap W_1 = \{a,e,f,o,r\}\\
\dom[V_{22}] &= W_2\cap W_2 = \{a,d,e,f,g,i,n,o,p,r,s,u,w,y\}\\
\dom[V_{23}] &= W_2\cap W_3 = \{a,d,e,f,g,n,o,r,y\}\\
\dom[V_{31}] &= W_3\cap W_1 = \{a,e,f,l,o,r,t\}\\
\dom[V_{32}] &= W_3\cap W_2 = \{a,d,e,f,g,n,o,r,y\}\\
\dom[V_{33}] &= W_3\cap W_3 = \{a,c,d,e,f,g,h,k,l,m,n,o,r,t,y\}
\end{aligned}\]
\textbf{限制条件}$C(X,Y,Z)$代表$\overline{XYZ}$构成的字符串在单词列表中,所有限制如下:
\[\begin{aligned}
    & C(V_{11},V_{12},V_{13}),C(V_{21},V_{22},V_{23}),C(V_{31},V_{32},V_{33})\\
    & C(V_{11},V_{21},V_{31}),C(V_{12},V_{22},V_{32}),C(V_{13},V_{23},V_{33})
\end{aligned}\]

\item [(b)] 设\textbf{变量}$V_1,V_2,\ldots,V_k$为独立集(independent set)中的$k$个元素,每一个$V_i$的\textbf{论域}均为图$\sG$的所有顶点$\sV$,\textbf{限制条件}为
\[C(V_1,V_2,\ldots,V_k):=\;(\forall V_i,V_j(i,j\in\{1,\ldots,k\}\land i\ne j):\;V_i\text{不与}V_j\text{相邻})\]

\item [(c)] 设\textbf{变量}为$V_S,V_E,V_N,V_D,V_M,V_O,V_R,V_Y$,每一个变量的\textbf{论域}均为\{0,1,\ldots,9\},\textbf{约束条件}为
\[C(V_S,V_E,V_N,V_D,V_M,V_O,V_R,V_Y)\]
满足
\[1000(V_S+V_M)+100(V_E+V_O)+10(V_N+V_R)+(V_D+V_E)=10000V_M+1000V_O+100V_N+10V_E+V_Y\]
\end{itemize}
\end{answer}
\begin{flushleft}
\textemph{
Score: 13/15
\begin{itemize}
\item (b)和(c)的Constraints都缺少allDifferent(Vars)
\end{itemize}
}
\end{flushleft}

\newpage
\section{Q2 - FC \& GAC}
\begin{question}\normalfont
    Consider a scheduling problem, where there are five activities to be scheduled in four time slots. Suppose we represent the activities by the variables $A, B,  C, D, E$, where the domain of each variable is $\{1,2,3,4\}$ and the constraints are $A > D$, $D > E$, $C \ne A$, $C > E$, $C \ne D$, $B \geq A$, $B \ne C$, and $C \ne D + 1$.
    \begin{itemize}
    \item [(a)] Find the first solution by using the Forward Checking algorithm with the MRV heuristics that we always instantiate next that variable with smallest remaining number of elements in its domain, breaking ties in alphabetic order. Assign values in the current domain of each variable in increasing order. At each node indicate
    \begin{itemize}
        \item [i.] The variable being instantiated and the value being assigned to it.
        \item [ii.] The CurDom for each variable.
        \item [iii.] Mark any node with an empty CurDom with DWO.
    \end{itemize}
    \item [(b)] Enforce GAC on the constraints, and give the resultant variable domains. You should show which values of a domain are removed at each step, and which arc is responsible for removing the value. Then use the GAC algorithm to find the first solution.
    \end{itemize}
\end{question}
\begin{answer}
\begin{itemize}
\item [(a)] 对限制条件重新排序,有
\begin{itemize}
    \item A: $C(A,B),C(A,C),C(A,D)$
    \item B: $C(B,A),C(B,C)$
    \item C: $C(C,A),C(C,B),C_1(C,D),C_2(C,D),C(C,E)$
    \item D: $C(D,A),C_1(D,C),C_2(D,C),C(D,E)$
    \item E: $C(E,C),C(E,D)$
\end{itemize}
搜索过程如下,用中括号表示论域,没有中括号则代表赋值。
注意搜索限制按照字母序进行搜索,因此可能出现未搜索到对应限制(未进行论域删减),就提前终止了。
删除线划掉的代表论域已经失效。

\begin{longtable}{|c|c|c|c|c|c|}\hline
    Depth & A & B & C & D & E \\\hline
    0 & [1, 2, 3, 4] & [1, 2, 3, 4] & [1, 2, 3, 4] & [1, 2, 3, 4] & [1, 2, 3, 4]\\\hline
    1 & 1 & \sout{[1, 2, 3, 4]} & \sout{[1, 2, 3, 4]} & DWO & \sout{[1, 2, 3, 4]}\\ \hline
    1 & 2 & [2, 3, 4] & [1, 3, 4] & [1] & [1, 2, 3, 4]\\ \hline
    2 & 2 & \sout{[2, 3, 4]} & \sout{[1, 3, 4]} & 1 & DWO\\ \hline
    1 & 3 & [3, 4] & [1, 2, 4] & [1, 2] & [1, 2, 3, 4]\\ \hline
    2 & 3 & 3 & [1, 2, 4] & [1, 2] & [1, 2, 3, 4]\\ \hline
    3 & 3 & 3 & \sout{[1, 2, 4]} & 1 & DWO\\ \hline
    3 & 3 & 3 & [1, 4] & 2 & [1]\\ \hline
    4 & 3 & 3 & [4] & 2 & 1\\ \hline
    5 & 3 & 3 & 4 & 2 & 1\\ \hline
\end{longtable}

最终得到解为$A=3,B=3,C=4,D=2,E=1$。
\item [(b)]
搜索及剪枝过程如下,其中首列为搜索深度或剪枝的条件(违反一致性的边(arc))。
后面的记号与(a)题相同,没有加方括号的为赋值,加了方括号的为论域,每轮GAC检测删除的值都已用删除线标出。
注意这里合并了连续对同一变量的论域进行删减的情况。

\begin{longtable}{|c|c|c|c|c|c|}\hline
    Depth / Constraints & A & B & C & D & E\\\hline
    0 & [1, 2, 3, 4] & [1, 2, 3, 4] & [1, 2, 3, 4] & [1, 2, 3, 4] & [1, 2, 3, 4]\\ \hline
    1 & 1, \color{red}{[1]} & [1, 2, 3, 4] & [1, 2, 3, 4] & [1, 2, 3, 4] & [1, 2, 3, 4]\\ \hline
    $a > d$ & 1, \color{red}{[\cancel{1}] (DWO)} & [1, 2, 3, 4] & [1, 2, 3, 4] & [1, 2, 3, 4] & [1, 2, 3, 4]\\ \hline
    1 & 2, \color{red}{[2]} & [1, 2, 3, 4] & [1, 2, 3, 4] & [1, 2, 3, 4] & [1, 2, 3, 4]\\ \hline
    $a \leq b$ & 2, [2] & \color{red}{[\cancel{1}, 2, 3, 4]} & [1, 2, 3, 4] & [1, 2, 3, 4] & [1, 2, 3, 4]\\ \hline
    $a > d$ & 2, [2] & [2, 3, 4] & [1, 2, 3, 4] & \color{red}{[1, \cancel{2, 3, 4}]} & [1, 2, 3, 4]\\ \hline
    $a \ne c$ & 2, [2] & [2, 3, 4] & \color{red}{[1, \cancel{2}, 3, 4]} & [1] & [1, 2, 3, 4]\\ \hline
    $d > e$ & 2, [2] & [2, 3, 4] & [1, 3, 4] & \color{red}{[\cancel{1}] (DWO)} & [1, 2, 3, 4]\\ \hline
    1 & 3, \color{red}{[3]} & [1, 2, 3, 4] & [1, 2, 3, 4] & [1, 2, 3, 4] & [1, 2, 3, 4]\\ \hline
    $a \leq b$ & 3, [3] & \color{red}{[\cancel{1, 2}, 3, 4]} & [1, 2, 3, 4] & [1, 2, 3, 4] & [1, 2, 3, 4]\\ \hline
    $a > d$ & 3, [3] & [3, 4] & [1, 2, 3, 4] & \color{red}{[1, 2, \cancel{3, 4}]} & [1, 2, 3, 4]\\ \hline
    $a \ne c$ & 3, [3] & [3, 4] & \color{red}{[1, 2, \cancel{3}, 4]} & [1, 2] & [1, 2, 3, 4]\\ \hline
    $d > e$ & 3, [3] & [3, 4] & [1, 2, 4] & \color{red}{[\cancel{1}, 2]} & [1, 2, 3, 4]\\ \hline
    $d > e$ & 3, [3] & [3, 4] & [1, 2, 4] & [2] & \color{red}{[1, \cancel{2, 3, 4}]}\\ \hline
    $d \ne c$ & 3, [3] & [3, 4] & \color{red}{[1, \cancel{2}, 4]} & [2] & [1]\\ \hline
    $c > e$ & 3, [3] & [3, 4] & \color{red}{[\cancel{1}, 4]} & [2] & [1]\\ \hline
    $c \ne b$ & 3, [3] & \color{red}{[3, \cancel{4}]} & [4] & [2] & [1]\\ \hline
    2 & 3, [3] & 3, \color{red}{[3]} & [4] & [2] & [1]\\ \hline
    3 & 3, [3] & 3, [3] & 4, \color{red}{[4]} & [2] & [1]\\ \hline
    4 & 3, [3] & 3, [3] & 4, [4] & 2, \color{red}{[2]} & [1]\\ \hline
    5 & 3, [3] & 3, [3] & 4, [4] & 2, [2] & 1, \color{red}{[1]}\\ \hline
\end{longtable}

可以看出GAC算法对于剪枝非常高效,使得最后只扩展了六个结点就搜索到了解$A=3,B=3,C=4,D=2,E=1$。
\end{itemize}
\end{answer}
\begin{flushleft}
\textemph{
Score: 35/35
}
\end{flushleft}

\newpage
\section{Q3 - 知识表示与推理(归结)}
\begin{question}\normalfont
    Determine whether the following sentence is valid using resolution:
    \[(\exists x\forall y P(x,y)\lor\exists x\forall y Q(x,y))\to
    \exists x\forall y(P(x,y)\lor Q(x,y))\]
\end{question}
\begin{answer}
    按照子句形式的转换顺序,有
    \begin{itemize}
        \item 改写为归结的等价形式(消除蕴含)
        \[(\exists x\forall y P(x,y)\lor\exists x\forall y Q(x,y))\land\lnot
        (\exists x\forall y(P(x,y)\lor Q(x,y)))\]
        \item 否定内移
        \[(\exists x\forall y P(x,y)\lor\exists x\forall y Q(x,y))\land
        (\forall x\exists y (\lnot P(x,y)\land \lnot Q(x,y)))\]
        \item 标准化变量
        \[(\exists x\forall y P(x,y)\lor\exists z\forall w Q(z,w))\land
        (\forall a\exists b (\lnot P(a,b)\land \lnot Q(a,b)))\]
        \item Skolemize
        \[(\forall y P(x,y)\lor \forall w Q(z,w))\land
        (\forall a (\lnot P(a,f(a))\land \lnot Q(a,f(a))))\]
        \item 改为前缀形式
        \[\forall y\forall w\forall a ((P(x,y)\lor Q(z,w))\land
        (\lnot P(a,f(a))\land \lnot Q(a,f(a))))\]
        \item 变成子句形式
        \begin{itemize}
            \item [1.] $P(x,y),Q(z,w)$
            \item [2.] $\lnot P(a,f(a))$
            \item [3.] $\lnot Q(a,f(a))$
        \end{itemize}
    \end{itemize}
    归结过程如下
    \begin{enumerate}
        \item [4.] $R[1a,2]\{x=a,y=f(a)\}$:$(Q(z,w))$
        \item [5.] $R[3,4]\{z=a,w=f(a)\}$:$()$
    \end{enumerate}
    总结即当$x=a,y=f(a),z=a,w=f(a)$时得到归结结果,题目中的式子成立。
\end{answer}
\begin{flushleft}
\textemph{
Score: 13/15
\begin{itemize}
\item 在Skolemize表示中,$x$和$z$是常量,$a$是变量,故在归结的时候应写为$\{a=x,a=z\}$
\end{itemize}
}
\end{flushleft}

\newpage
\section{Q4 - 知识表示与推理(归结)}
\begin{question}\normalfont
    Victor has been murdered, and Arthur, Bertram, and Carleton are the only suspects (meaning exactly one of them is the murderer). Arthur says that Bertramwas the victim's friend, but that Carleton hated the victim. Bertram says that he was out of town the day of the murder, and besides, he didn't even know the guy. Carleton says that he saw Arthur and Bertram with the victim just before the murder. You may assume that everyone – except possibly for the murderer – is telling the truth.
    \begin{itemize}
    \item [(a)] Use Resolution to find the murderer. In other words, formalize the facts as a set of clauses, prove that there is a murderer, and extract his identity from the derivation.
    \item [(b)] Suppose we discover that we were wrong – we cannot assume that there was only a single murderer (there may have been a conspiracy). Show that in this case the facts do not support anyone's guilt. In other words, for each suspect, present a logical interpretation that supports all the facts but where that suspect is innocent and the other two are guilty.
    \end{itemize}
\end{question}
\begin{answer}
\begin{itemize}
    \item [(a)] 凶手既可以说真话也可以说假话,而不是凶手则一定说真话,则凶手与他说的话之间呈\emph{或}的关系。
    简记Arthur, Bertram, Carleton, Victor为A, B, C, V,并设谓词$M(x)$代表$x$为凶手,$F(x)$代表$x$为V的朋友,$O(x)$代表$x$有不在场证明,$x\in D=\{A,B,C\}$。
    则由题目信息,可给出下列公式
    \[\begin{aligned}
        & M(A)\lor(F(B)\land\lnot F(C))\\
        & M(B)\lor(O(B)\land\lnot F(B))\\
        & M(C)\lor(\lnot O(A)\land\lnot O(B))
    \end{aligned}\]
    并且仅仅只有一个凶手,故
    \[M(A)\oplus M(B)\oplus M(C)\]
    其中$x \oplus y:=(x\lor y)\land(\lnot x\lor\lnot y)$代表异或。
    全部写成子句形式,我们有
    \begin{enumerate}
        \item $(M(A),F(B))$
        \item $(M(A),\lnot F(C))$
        \item $(M(B),O(B))$
        \item $(M(B),\lnot F(B))$
        \item $(M(C),\lnot O(A))$
        \item $(M(C),\lnot O(B))$
        \item $(M(A), M(B), M(C))$
        \item $(\lnot M(A), \lnot M(B))$
        \item $(\lnot M(B), \lnot M(C))$
        \item $(\lnot M(C), \lnot M(A))$
    \end{enumerate}
    询问为$\exists xM(x)$,为提取答案有下列子句
    \begin{flushleft}
        11. $(\lnot M(x),Answer(x))$
    \end{flushleft}
    做以下归约
    \begin{enumerate}
        \item[12.] $R[1,4]:\;(M(A),M(B))$
        \item[13.] $R[3,6]:\;(M(B),M(C))$
        \item[14.] $R[10,12]:\;(\lnot M(C),M(B))$
        \item[15.] $R[13,14]:\;(M(B))$
        \item[16.] $R[11,15]\{x=B\}:\;(Answer(B))$
    \end{enumerate}
    所以得到Bertram为凶手。
    \item [(b)] 如果凶手可能不止一个,则只有1-7条子句仍继续符合。下面给出可能的解释:
    \begin{itemize}
        \item 当$A$为清白,$B,C$为凶手时:$A$说真话,$B$是朋友,$C$不是朋友;$B$说假话,$B$当天在场且$B$是朋友;$C$可说真话可说假话。
        形式化的解释$\Im$如下,用\cmark代表True,用\xmark代表False。
        \begin{center}
            \begin{tabular}{|c|c|c|c|}\hline
                 & A & B & C\\\hline
                $I(M)$ & \xmark & \cmark & \cmark\\\hline
                $I(O)$ & \xmark & \xmark & \xmark\\\hline
                $I(F)$ & \cmark & \cmark & \xmark\\\hline
            \end{tabular}
        \end{center}
        \item 当$B$为清白,$A,C$为凶手时:$B$说真话,$B$不在场,且$B$不是朋友;$A$说假话,$B$不是朋友,$C$可以是朋友也可以不是;$C$说假话,$B$不在场,$A$可在场可不在场。
        一种可能解释如下
        \begin{center}
            \begin{tabular}{|c|c|c|c|}\hline
                 & A & B & C\\\hline
                $I(M)$ & \cmark & \xmark & \cmark\\\hline
                $I(O)$ & \xmark & \cmark & \xmark\\\hline
                $I(F)$ & \cmark & \xmark & \cmark\\\hline
            \end{tabular}
        \end{center}
        \item 当$C$为清白,$A,B$为凶手时:$C$说真话,$A$在场,$B$在场;$B$说假话,$B$在场,但$B$是否是朋友不确定;$A$可说真话也可说假话,$B,C$是否朋友不确定。
        一种可能解释如下
        \begin{center}
            \begin{tabular}{|c|c|c|c|}\hline
                 & A & B & C\\\hline
                $I(M)$ & \cmark & \cmark & \xmark\\\hline
                $I(O)$ & \xmark & \xmark & \xmark\\\hline
                $I(F)$ & \cmark & \cmark & \xmark\\\hline
            \end{tabular}
        \end{center}
    \end{itemize}
\end{itemize}
\end{answer}
\begin{flushleft}
\textemph{
Score: 35/35
}
\end{flushleft}

\end{document}

% Due: 2019.10.23