\documentclass[logo,reportComp]{thesis}
\usepackage[cpp,pseudo]{mypackage}
\usepackage{forest}

\usetikzlibrary{automata,backgrounds,fit,shapes,positioning}

\tikzset{->, % makes the edges directed
>=stealth, % makes the arrow heads bold
node distance=2.5cm, % specifies the minimum distance between two nodes. Change if necessary.
every state/.style={thick, fill=gray!10}, % sets the properties for each 'state' node
}

\forestset{
sn edges/.style={for tree={edge={-}}}
}

\title{编译原理作业九}
\subtitle{}
\school{数据科学与计算机学院}
\author{陈鸿峥}
\classname{17大数据与人工智能}
\stunum{17341015}
\headercontext{编译原理作业}


\begin{document}

\maketitle
\renewcommand\arraystretch{0.8}
\begin{question}
令\verb'a'表示一个$2\times 3$的整型数组,\verb'b'表示一个$3\times 4$的整型数组.
假定一个整数的宽度为$4$.
试使用课本图6.22的翻译模式,翻译赋值语句\verb'x=a[i][j]+b[i][j]'.
提示:参考课本例6.12.
\end{question}
\begin{answer}
带注释的分析树如下
\begin{center}
\scalebox{0.8}{
\begin{forest}
sn edges
[{$S$}
	[{$x$}]
	[{$=$}]
	[{$E.addr=t_9$}
		[{$E.addr=t_4$}
			[{\begin{tabular}{l}$L.array=a$\\ $L.type=int$\\ $L.addr=t_3$\end{tabular}}
				[{\begin{tabular}{l}$L.array=a$\\ $L.type=array(3,int)$\\ $L.addr=t_1$\end{tabular}}
					[{\begin{tabular}{l}$a.type=$\\$array(2,array(3,int))$\end{tabular}}]
					[{$[$}]
					[{$E.addr=i$}
						[{$i$}]
					]
					[{$]$}]
				]
				[{$[$}]
				[{$E.addr=j$}
					[{$j$}]
				]
				[{$]$}]
			]
		]
		[{$+$}]
		[{$E.addr=t_8$}
			[{\begin{tabular}{l}$L.array=b$\\ $L.type=int$\\ $L.addr=t_7$\end{tabular}}
				[{\begin{tabular}{l}$L.array=b$\\ $L.type=array(4,int)$\\ $L.addr=t_5$\end{tabular}}
					[{\begin{tabular}{l}$b.type=$\\$array(3,array(4,int))$\end{tabular}}]
					[{$[$}]
					[{$E.addr=i$}
						[{$i$}]
					]
					[{$]$}]
				]
				[{$[$}]
				[{$E.addr=j$}
					[{$j$}]
				]
				[{$]$}]
			]
		]
	]
]
\end{forest}
}
\end{center}

生成的三地址码如下
\begin{center}
\begin{tabular}{l}
$t_1 = i * 12$\\
$t_2 = j * 4$\\
$t_3 = t_1 + t_2$\\
$t_4 = a[t_3]$\\
$t_5 = i * 16$\\
$t_6 = j * 4$\\
$t_7 = t_5 + t_6$\\
$t_8 = b[t_7]$\\
$t_9 = t_4 + t_8$\\
$x   = t_9$
\end{tabular}
\end{center}
\end{answer}

\begin{question}[课本Exercise 6.6.1 a)]
在图 6-36 的语法制导定义中添加处理下列控制流构造的规则:
\begin{itemize}
\item 一个 \verb'repeat' 语句:\verb'repeat S while B'
\end{itemize}
\end{question}
\begin{answer}
语法制导定义如下
\begin{center}
\begin{tabular}{l|l}
Production & Semantic Rules\\\hline
\verb'S -> repeat S1 while B' &
\begin{tabular}{l}
\verb'S1.next = newlabel()'\\
\verb'B.true = newlabel()'\\
\verb'B.false = S.next'\\
\verb'S.code = label(B.true) || S1.code'\\
\verb'        || label(S1.next) || B.code'
\end{tabular}
\end{tabular}
\end{center}
\end{answer}

\end{document}