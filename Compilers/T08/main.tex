\documentclass[logo,reportComp]{thesis}
\usepackage[cpp,pseudo]{mypackage}
\usepackage{forest}

\usetikzlibrary{automata,backgrounds,fit,shapes,positioning}

\tikzset{->, % makes the edges directed
>=stealth, % makes the arrow heads bold
node distance=2.5cm, % specifies the minimum distance between two nodes. Change if necessary.
every state/.style={thick, fill=gray!10}, % sets the properties for each 'state' node
}

\forestset{
sn edges/.style={for tree={edge={-}}}
}

\title{编译原理作业八}
\subtitle{}
\school{数据科学与计算机学院}
\author{陈鸿峥}
\classname{17大数据与人工智能}
\stunum{17341015}
\headercontext{编译原理作业}

% June 16 -> June 21

\begin{document}

\maketitle

\begin{question}
考虑以下语法制导定义(Syntax Directed Definition):
\begin{center}
\begin{tabular}{|l|l|}\hline
\multicolumn{1}{|c|}{语法规则} & \multicolumn{1}{c|}{语义规则}\\\hline
$S \to ABCD$ & $S.val = A.val + B.val + C.val + D.val$\\\hline
$A \to gBa$ & $A.val = B.val * 5$\\\hline
$B \to B_1b$ & $B.val = B_1.val * 2$\\\hline
$B \to b$ & $B.val = 2$\\\hline
$C \to C_1c$ & $C.val = C_1.val * 3$\\\hline
$C \to c$ & $C.val = 3$\\\hline
$D \to d$ & $D.val = 1$\\\hline
\end{tabular}
\end{center}
对于输入串$gbbabbccd$构造带注释的分析树(annotated parse tree).
\end{question}
\begin{answer}
带注释的分析树如下
\begin{center}
\begin{forest}
sn edges
[{$S.val=20+4+9+1=34$}
	[{$A.val=4*5=20$}
		[{$\mathbf{g}$}]
		[{$B.val=2*2=4$}
			[{$B.val=2$}
				[{$\mathbf{b}$}]
			]
			[{$\mathbf{b}$}]
		]
		[{$\mathbf{a}$}]
	]
	[{$B.val=2*2=4$}
		[{$B.val=2$}
			[{$\mathbf{b}$}]
		]
		[{$\mathbf{b}$}]
	]
	[{$C.val=3*3=9$}
		[{$C.val=3$}
			[{$\mathbf{c}$}]
		]
		[{$\mathbf{c}$}]
	]
	[{$D.val=1$}
		[{$\mathbf{d}$}]
	]
]
\end{forest}
\end{center}
\end{answer}

\begin{question}
以下文法定义了二进制浮点数常量的语法规则:
\[\begin{aligned}
S &\to L.L \mid L\\
L &\to LB \mid B\\
B &\to 0 \mid 1
\end{aligned}\]
试给出一个$S$属性的语法制导定义,其作用是求出该二进制浮点数的十进制值,并存放在开始符号$S$相关联的一个综合属性$value$中。
例如,对于输入串$101.101$,$S$的$value$属性值结果应该是$5.625$。
要求在编写语法制导定义时,不得改写文法!
\end{question}
\begin{answer}
语法制导定义如下,按照二进制和十进制转换规则编写即可。
\begin{center}
\begin{tabular}{|l|l|}\hline
\multicolumn{1}{|c|}{语法规则} & \multicolumn{1}{c|}{语义规则}\\\hline
$S\to L_1.L_2$ & $S.value=L_1.value+L_2.value * 2^{-L_2.length}$\\\hline
$S\to L$ & $S.value=L.value$\\\hline
$L\to L_1B$ &
\begin{tabular}{l}
$L.value = L_1.value*2 + B.value$\\
$L.length = L_1.length + 1$
\end{tabular}\\\hline
$L\to B$ &
\begin{tabular}{l}
$L.value = B.value$\\
$L.length = 1$
\end{tabular}\\\hline
$B\to 0$ & $B.value=0$\\\hline
$B\to 1$ & $B.value=1$\\\hline
\end{tabular}
\end{center}
\end{answer}

\end{document}