\documentclass[logo,reportComp]{thesis}
\usepackage[cpp,pseudo]{mypackage}
\usepackage{forest}

\usetikzlibrary{automata,backgrounds,fit,shapes,positioning}

\tikzset{->, % makes the edges directed
>=stealth, % makes the arrow heads bold
node distance=2.5cm, % specifies the minimum distance between two nodes. Change if necessary.
every state/.style={thick, fill=gray!10}, % sets the properties for each 'state' node
}

\title{编译原理作业五}
\subtitle{}
\school{数据科学与计算机学院}
\author{陈鸿峥}
\classname{17大数据与人工智能}
\stunum{17341015}
\headercontext{编译原理作业}

% May 26 -> May 31

\begin{document}

\maketitle

\begin{question}
考虑以下文法:
\[\begin{aligned}
S &\to aTUV \mid bV\\
T &\to U \mid UU\\
U &\to \epsilon \mid bV\\
V &\to \epsilon \mid cV
\end{aligned}\]
写出每个非终端符号的FIRST集和FOLLOW集.
\end{question}
\begin{answer}
由文法可得
\[\begin{array}{rlrl}
FIRST(S) &= \{a,b\} & FOLLOW(S) &= \{\$\}\\
FIRST(T) &= \{\epsilon,b\} & FOLLOW(T) &= \{b,c,\$\}\\
FIRST(U) &= \{\epsilon,b\} & FOLLOW(U) &= \{b,c,\$\}\\
FIRST(V) &= \{\epsilon,c\} & FOLLOW(V) &= \{b,c,\$\}\\
\end{array}\]
\end{answer}

\begin{question}
考虑以下文法:
\[\begin{aligned}
S &\to (L) \mid a\\
L &\to L, S \mid S
\end{aligned}\]
\begin{enumerate}
	\item 消除文法的左递归.
	\item 构造文法的LL(1)分析表.
	\item 对于句子$(a, (a, a))$,给出语法分析的详细过程(参照课本228页的图4.21).
\end{enumerate}
\end{question}
\begin{answer}
\begin{enumerate}
	\item 如下
	\[\begin{aligned}
	S &\to (L)\mid a\\
	L &\to SL'\\
	L' &\to , SL'\mid\epsilon
	\end{aligned}\]
	\item 先求出FIRST集和FOLLOW集(由于文法中存在逗号,故将字符用引号括起来以示区分)
	\[\begin{array}{rlrl}
	FIRST(S) &= \{'(','a'\} & FOLLOW(S) &= \{'\$',')'\}\\
	FIRST(L) &= \{'(','a'\} & FOLLOW(L) &= \{')'\}\\
	FIRST(L') &= \{',',\epsilon\} & FOLLOW(L') &= \{')'\}
	\end{array}\]
	LL(1)分析表如下,其中第一列为非终端符号,第一行为输入符号.
	\begin{center}
	\begin{tabular}{|c|c|c|c|c|c|}\hline
	 & $($ & $)$ & $a$ & $,$ & \$\\\hline
	S & $S\to(L)$ & & $S\to a$ & &\\\hline
	L & $L\to SL'$ & & $L\to SL'$ & &\\\hline
	L' & & $L'\to\epsilon$ & & $L'\to,SL'$ & \\\hline
	\end{tabular}
	\end{center}
	\item 语法分析过程如下
	\begin{center}
	\begin{tabular}{|l|r|r|l|}\hline
	Matched & Stack & Input & Action\\\hline
	 & $S\$$ & $(a,(a,a))\$$ & \\\hline
	 & $(L)\$$ & $(a,(a,a))\$$ & output $S\to (L)$\\\hline
	$($ & $L)\$$ & $a,(a,a))\$$ & \\\hline
	$($ & $SL')\$$ & $a,(a,a))\$$ & output $L\to SL'$\\\hline
	$($ & $aL')\$$ & $a,(a,a))\$$ & output $S\to a$\\\hline
	$(a$ & $L')\$$ & $,(a,a))\$$ & \\\hline
	$(a$ & $,SL')\$$ & $,(a,a))\$$ & output $L'\to ,SL'$\\\hline
	$(a,$ & $SL')\$$ & $(a,a))\$$ & \\\hline
	$(a,$ & $(L)L')\$$ & $(a,a))\$$ & output $S\to (L)$\\\hline
	$(a,($ & $L)L')\$$ & $a,a))\$$ & \\\hline
	$(a,($ & $SL')L')\$$ & $a,a))\$$ & output $L\to SL'$\\\hline
	$(a,($ & $aL')L')\$$ & $a,a))\$$ & output $S\to a$\\\hline
	$(a,(a$ & $L')L')\$$ & $,a))\$$ & \\\hline
	$(a,(a$ & $,SL')L')\$$ & $,a))\$$ & output $L'\to,SL'$\\\hline
	$(a,(a,$ & $SL')L')\$$ & $a))\$$ & \\\hline
	$(a,(a,$ & $aL')L')\$$ & $a))\$$ & output $S\to a$\\\hline
	$(a,(a,a$ & $L')L')\$$ & $))\$$ & \\\hline
	$(a,(a,a$ & $)L')\$$ & $))\$$ & output $L'\to\epsilon$\\\hline
	$(a,(a,a)$ & $L')\$$ & $)\$$ & \\\hline
	$(a,(a,a)$ & $)\$$ & $)\$$ & output $L'\to\epsilon$\\\hline
	$(a,(a,a))$ & $\$$ & $\$$ & \\\hline
	\end{tabular}
	\end{center}
\end{enumerate}
\end{answer}

\begin{question}
考虑以下文法:
\[S \to aSbS \mid bSaS \mid \epsilon\]
这一文法是否是LL(1)文法?给出理由.
\end{question}
\begin{answer}
这一文法不是LL(1)文法.
$S$有产生式$S\to\epsilon$,但可求得$FIRST(S)=\{a,b,\epsilon\}$及$FOLLOW(S)=\{a,b,\$\}$.
因为$FIRST(S)\cap FOLLOW(S)\ne\varnothing$,故不是LL(1)文法.
\end{answer}

\end{document}