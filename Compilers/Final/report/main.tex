\documentclass[logo,reportComp]{thesis}
\usepackage[cpp,linenum]{mypackage}

\title{编译原理期末大作业}
\subtitle{正则表达式等价性}
\school{数据科学与计算机学院}
\author{陈鸿峥}
\classname{17大数据与人工智能}
\stunum{17341015}
\headercontext{编译原理期末大作业}

\begin{document}

\maketitle

\section{实验目的}
对于两个正则表达式$r$和$s$,判断这两个正则表达式的关系。
正则表达式$r$和$s$的关系有4种:
\begin{enumerate}
\item $r$和$s$等价,即$r$描述的语言和$s$描述的语言相等;
\item $r$描述的语言是$s$描述的语言的真子集;
\item $s$描述的语言是$r$描述的语言的真子集;
\item 非上述情况。
\end{enumerate}
正则表达式的字符集为小写字母\verb'a'-\verb'z',\verb'|'号表示或者,\verb'*'号表示闭包,\verb'?'表示出现$0$或$1$次,$+$表示至少出现一次,大写字母$E$表示epsilon(空串)。

编写一个C++程序,实现上述功能。

\textbf{输入格式:}\par
第一行是测试数的组数$T$。
接下来的$T$行,每行是两个正则表达式$r$和$s$,每个正则表达式只含\verb'a'-\verb'z',\verb'|',\verb'*',\verb'?',\verb'+',\verb'(',\verb')',\verb'E'。
两个正则表达式之间用空格分开。

\textbf{输出格式:}\par
输出有$T$行。
对于每组数据,如果$r$和$s$等价,输出\verb'=';
如果$r$是$s$的真子集,输出\verb'<';
如果$s$是$r$的真子集,输出\verb'>';
非上述情况,输出\verb'!'。

\textbf{提交内容:}
\begin{enumerate}
	\item 能在Linux下或在Windows的Dev C++下编译运行的C++源程序;
	\item 实验报告,包括算法描述,和你的测试用例,及测试结果。
\end{enumerate}

提交截止日期:7.25

\section{实验结果}


\end{document}