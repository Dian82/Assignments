\documentclass[logo,reportComp]{thesis}
\usepackage[cpp,pseudo]{mypackage}

\title{操作系统原理实验报告}
\subtitle{实验三:开发独立内核的操作系统}
\school{数据科学与计算机学院}
\author{陈鸿峥}
\classname{17大数据与人工智能}
\stunum{17341015}
\headercontext{操作系统原理实验报告}
\authorremark{本实验报告用\LaTeX撰写,创建时间:\builddate\today}


\begin{document}

\maketitle

\section{实验目的}
% 牵引车(实验一)->火车头(实验三)->车厢(实验二)
\begin{itemize}
    \item 掌握C语言与汇编混合编程的方法
    \item 分离引导程序和内核,学会用引导程序引导系统内核
    \item 为操作系统提供批处理能力
\end{itemize}

\section{实验要求}
% 实验目的和实验要求由老师提供实验项目文档中获取
\begin{itemize}
    \item 将实验二的原型操作系统分离为引导程序和MYOS内核,由引导程序加载内核,用C和汇编实现操作系统内核
    \item 扩展内核汇编代码,增加一些有用的输入输出函数,供C模块中调用
    \item 提供用户程序返回内核的一种解决方案
    \item 在内核的C模块中实现增加批处理能力
    \begin{itemize}
        \item 在磁盘上建立一个表,记录用户程序的存储安排
        \item 可以在控制台命令查到用户程序的信息,如程序名、字节数、在磁盘映像文件中的位置等
        \item 设计一种命令,命令中可加载多个用户程序,依次执行,并能在控制台发出命令
        \item 在引导系统前,将一组命令存放在磁盘映像中,系统可以解释执行
    \end{itemize}
\end{itemize}

\section{实验环境}
% 包括:硬件或虚拟机配置方法、软件工具与作用、方案的思想、相关原理、程序流程、算法和数据结构、程序关键模块,结合代码与程序中的位置位置进行解释。不得抄袭,否则按作弊处理。
% 实验方案包括相关基础原理、实验工具和环境、程序流程和算法思想、数据结构与程序模块功能说明,代码文档组成说明等
具体环境选择原因已在实验一报告中说明。
\begin{itemize}
	\item Windows 10系统 + Ubuntu 18.04(LTS)子系统
	\item gcc 7.3.0 + nasm 2.13.02 + gdb
	\item Oracle VM VirtualBox 5.2.8
	\item Sublime Text 3
\end{itemize}

虚拟机配置:内存4M,无硬盘,1.44M虚拟软盘引导。

\section{实验方案}
% 包括:主要工具安装使用过程及截图结果、程序过程中的操作步骤、测试数据、输入及输出说明、遇到的问题及解决情况、关键功能或操作的截图结果。不得抄袭,否则按作弊处理。


\section{实验结果}


\section{实验总结}
% 每人必需写一段,文字不少于500字,可以写心得体会、问题讨论与思考、新的设想、感言总结或提出建议等等。不得抄袭,否则按作弊处理。



\section{参考资料}
\begin{enumerate}
	\item 李忠,王晓波,余洁,《x86汇编语言-从实模式到保护模式》,电子工业出版社,2013
\end{enumerate}

\appendix
\appendixconfig
\section{程序清单}
\label{sec:code}

\begin{lstlisting}[language={[x86masm]Assembler}]

\end{lstlisting}

\section{附件文件说明}
\begin{center}
\begin{tabular}{|c|l|l|}\hline
序号 & 文件 & 描述 \\\hline
\end{tabular}
\end{center}

\end{document}

% 实验提交内容
% 实验报告:电子版(Word2003的DOC格式或PDF格式)
% 原程序文件及可执行代码程序文件
% 测试输入数据文件和输出数据文件
% 虚拟机软盘映像文件

% 基础实验项目5个和扩展实验7个
% 实验项目,迟交影响成绩评价!
% 工具与环境可由选择,开发新型工具或优化一套开发环境都可加分!
% 一系列基础实验项目必须连续完成,当前项目只能在前一个项目的基础上进行,体现出前后的进化关系,否则要被约谈,证明没有抄袭行为!
% 一个项目可提交多个改进的版本,实现新功能和个性化特征都有利于提高相应项目的成绩。
% 实验项目提交内容用winrar工具整体压缩打包,统一格式命名为:
%    <学号>+<姓名>+<实验项目号>+<版本号>.rar
%    姓名(学号)实验NvX.zip
%    实验报告、项目文件夹、映像文件
%    ftp://172.18.216.232 sysuac 下周六23:59

% 免考
% 条件:实验1~6全部评价AAAAB+B+或相当
% 最终成绩可能范围:75分以上