\documentclass[logo,reportComp]{thesis}
\usepackage{mypackage}

\title{机器学习与数据挖掘课堂作业}
\subtitle{主成分分析}
\school{数据科学与计算机学院}
\author{陈鸿峥}
\classname{17大数据与人工智能}
\stunum{17341015}
\headercontext{机器学习与数据挖掘课堂作业}

\begin{document}

\maketitle

\begin{question}
PCA仅需保留$W$与样本的均值向量即可通过简单的向量减法和矩阵-向量乘法将新样本投影至低维空间中。如何做?
\end{question}
\begin{answer}
设样本为$\vx_i$,样本均值为$\bar{\vx}$,则由求解出的PCA矩阵$W$和线性变换,可得到降维后样本向量$\vx_i'$
\[\vx_i'=W^\T(\vx_i-\bar{\vx})\]
注意这里利用了样本均值进行了与PCA同样的中心化操作。
\end{answer}

\end{document}