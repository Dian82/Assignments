\documentclass[logo,reportComp]{thesis}
\usepackage{mypackage}

\title{机器学习与数据挖掘课堂作业}
\subtitle{集成学习}
\school{数据科学与计算机学院}
\author{陈鸿峥}
\classname{17大数据与人工智能}
\stunum{17341015}
\headercontext{机器学习与数据挖掘课堂作业}

\begin{document}

\maketitle

\begin{question}
\normalfont By combining bagging and boosting strategies, design a more effective ensemble framework.
\end{question}
\begin{answer}
由于bagging过程中会对训练集进行划分,然后训练每一基学习器,而每一基学习器都是相互独立的,因此可以用boosting的方法训练bagging的基学习器。
当boosting训练出的基学习器足够强时(比如比单一的决策树要好),那利用bagging将这些基学习器整合在一起,性能也会更优。
又由于bagging中每个基学习器的训练都可以并行完成,因此即使在基学习器中使用了boosting,依然可以保持较快的速度。
\end{answer}

\end{document}