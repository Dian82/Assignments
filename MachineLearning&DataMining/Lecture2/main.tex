\documentclass[logo,reportComp]{thesis}
\usepackage{mypackage}

\title{机器学习与数据挖掘作业一}
\subtitle{KNN算法}
\school{数据科学与计算机学院}
\author{陈鸿峥}
\classname{17大数据与人工智能}
\stunum{17341015}
\headercontext{机器学习与数据挖掘作业}

\begin{document}

\maketitle

\begin{question}
\normalfont How to measure the distance for categorical attributes with more than two values?\\
(Please consider from both sequential and non-sequential aspects.) % maybe ordered & unordered?
\end{question}
% https://www.quora.com/How-can-I-use-KNN-for-mixed-data-categorical-and-numerical
\begin{answer}
对于有序\footnote{问题中的sequential和non-sequential理解成有序(ordered)和无序(unordered)}的离散(categorical)属性,可以直接将每个类别定为一个数字。
举个例子,假设有一年龄属性分为三个类别,那么可以按下表进行数字化处理,因为年龄各类别间存在大小关系。
\begin{center}
\begin{tabular}{|c|c|c|c|}\hline
年龄 & $<18$岁 & $18\thicksim 30$岁 & $30$岁以上\\\hline
数值编码 & 0 & 1 & 2\\\hline
\end{tabular}
\end{center}

对于无序的离散属性,则可以采用独热码(one-hot encoding),以确保两两类别之间距离相同,因为类别之间并没有大小差异,而只是指代的东西不同。
比如,对于下列三种颜色,可以进行如下编码。
\begin{center}
\begin{tabular}{|c|c|c|c|}\hline
颜色 & 红 & 黄 & 蓝\\\hline
数值编码 & $(1,0,0)$ & $(0,1,0)$ & $(0,0,1)$\\\hline
\end{tabular}
\end{center}

当categorical属性转换成numerical属性后(离散转连续),那么就可以正常按照欧式距离、曼哈顿距离等进行度量。
\end{answer}

\begin{question}
\normalfont How to quickly retrieve $K$ nearest neighbors of a given query (sample)?
\end{question}
\begin{answer}
假设询问的点是$\vx_n$,特征的维度为$d(n>>d)$。
简单的算法是$\vx_n$与$\vx_i(i=1,\ldots,n-1)$分别算出距离后,对距离数组升序排序,取出最小的$k$个点,即为$k$个邻居。
算距离复杂度为$O(n)$,排序复杂度为$O(n\log n)$,顺序选邻居复杂度为$O(1)$,总的复杂度为$O(n\log n)$。

可以看到时间开销都花在排序上面,而事实上我们并不关心整体数组的序,而只关心最短距离的那$k$个点,故在$n>>k$的情况下,直接遍历$k$次数组每次记录最小距离,时间复杂度开销仅为$O(kn)$,关于$n$的线性复杂度。

通过维护一个大小为$k$优先队列/最大堆,我们可以做得更快。
每次插入元素时进行比较堆顶元素,如果当前距离小于堆顶,则将当前值放入堆中,这样子建堆的复杂度为$O(n\log k)$,$k$邻居即为该堆中的$k$个结点。

最后,目前比较常用的方法是KD树,用超平面对高维空间的点进行二划分(以均值为界),构建一棵平衡二叉树,建树复杂度为$O(dn\log n)$且可以重用,查询复杂度为$O(\log n)$。
因此KD树增大了预处理时间,但是却能够让KNN的查询效率大大提升。
\end{answer}
% https://sebastianraschka.com/pdf/lecture-notes/stat479fs18/02_knn_notes.pdf
% https://www.cnblogs.com/mantch/p/11287075.html

% \begin{thebibliography}{99}
% \bibitem{bib:item}
% \end{thebibliography}

\end{document}