\documentclass[logo,reportComp]{thesis}
\usepackage{mypackage}

\title{机器学习与数据挖掘课堂作业}
\subtitle{半监督学习}
\school{数据科学与计算机学院}
\author{陈鸿峥}
\classname{17大数据与人工智能}
\stunum{17341015}
\headercontext{机器学习与数据挖掘课堂作业}

\begin{document}

\maketitle

\begin{question}
分析生成式方法、半监督SVM、图半监督学习、半监督聚类等半监督方法的相同和不同之处。
\end{question}
\begin{answer}
这几种方法都属于半监督学习的范畴,也就是让学习器不依赖外界交互、自动利用未标记样本来提升学习性能。
他们都需要外部的信息或是一些预设的模型进行构造。

生成式方法会假设所有数据(无论是否有标记)都是由同一个潜在的模型生成的,而不同的模型假设将会产生不同的方法。
半监督SVM则是直接给定了“低密度分隔”的基本假设,试图找到能将两类有标记样本分开,且穿过数据低密度区域的划分超平面。
图半监督学习则给数据集中的样本强加了相似性/相关性,并依此构造了带“颜色”标注的图,通过“颜色”在图上扩散或传播的过程来实现学习。
半监督聚类充分利用了外部监督信息,包括同簇异簇的限制、少量的标记样本等,这些限制是与其他半监督方法的较大不同之处。
\end{answer}

\end{document}